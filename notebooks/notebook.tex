
% Default to the notebook output style

    


% Inherit from the specified cell style.




    
\documentclass[11pt]{article}

    
    
    \usepackage[T1]{fontenc}
    % Nicer default font (+ math font) than Computer Modern for most use cases
    \usepackage{mathpazo}

    % Basic figure setup, for now with no caption control since it's done
    % automatically by Pandoc (which extracts ![](path) syntax from Markdown).
    \usepackage{graphicx}
    % We will generate all images so they have a width \maxwidth. This means
    % that they will get their normal width if they fit onto the page, but
    % are scaled down if they would overflow the margins.
    \makeatletter
    \def\maxwidth{\ifdim\Gin@nat@width>\linewidth\linewidth
    \else\Gin@nat@width\fi}
    \makeatother
    \let\Oldincludegraphics\includegraphics
    % Set max figure width to be 80% of text width, for now hardcoded.
    \renewcommand{\includegraphics}[1]{\Oldincludegraphics[width=.8\maxwidth]{#1}}
    % Ensure that by default, figures have no caption (until we provide a
    % proper Figure object with a Caption API and a way to capture that
    % in the conversion process - todo).
    \usepackage{caption}
    \DeclareCaptionLabelFormat{nolabel}{}
    \captionsetup{labelformat=nolabel}

    \usepackage{adjustbox} % Used to constrain images to a maximum size 
    \usepackage{xcolor} % Allow colors to be defined
    \usepackage{enumerate} % Needed for markdown enumerations to work
    \usepackage{geometry} % Used to adjust the document margins
    \usepackage{amsmath} % Equations
    \usepackage{amssymb} % Equations
    \usepackage{textcomp} % defines textquotesingle
    % Hack from http://tex.stackexchange.com/a/47451/13684:
    \AtBeginDocument{%
        \def\PYZsq{\textquotesingle}% Upright quotes in Pygmentized code
    }
    \usepackage{upquote} % Upright quotes for verbatim code
    \usepackage{eurosym} % defines \euro
    \usepackage[mathletters]{ucs} % Extended unicode (utf-8) support
    \usepackage[utf8x]{inputenc} % Allow utf-8 characters in the tex document
    \usepackage{fancyvrb} % verbatim replacement that allows latex
    \usepackage{grffile} % extends the file name processing of package graphics 
                         % to support a larger range 
    % The hyperref package gives us a pdf with properly built
    % internal navigation ('pdf bookmarks' for the table of contents,
    % internal cross-reference links, web links for URLs, etc.)
    \usepackage{hyperref}
    \usepackage{longtable} % longtable support required by pandoc >1.10
    \usepackage{booktabs}  % table support for pandoc > 1.12.2
    \usepackage[inline]{enumitem} % IRkernel/repr support (it uses the enumerate* environment)
    \usepackage[normalem]{ulem} % ulem is needed to support strikethroughs (\sout)
                                % normalem makes italics be italics, not underlines
    

    
    
    % Colors for the hyperref package
    \definecolor{urlcolor}{rgb}{0,.145,.698}
    \definecolor{linkcolor}{rgb}{.71,0.21,0.01}
    \definecolor{citecolor}{rgb}{.12,.54,.11}

    % ANSI colors
    \definecolor{ansi-black}{HTML}{3E424D}
    \definecolor{ansi-black-intense}{HTML}{282C36}
    \definecolor{ansi-red}{HTML}{E75C58}
    \definecolor{ansi-red-intense}{HTML}{B22B31}
    \definecolor{ansi-green}{HTML}{00A250}
    \definecolor{ansi-green-intense}{HTML}{007427}
    \definecolor{ansi-yellow}{HTML}{DDB62B}
    \definecolor{ansi-yellow-intense}{HTML}{B27D12}
    \definecolor{ansi-blue}{HTML}{208FFB}
    \definecolor{ansi-blue-intense}{HTML}{0065CA}
    \definecolor{ansi-magenta}{HTML}{D160C4}
    \definecolor{ansi-magenta-intense}{HTML}{A03196}
    \definecolor{ansi-cyan}{HTML}{60C6C8}
    \definecolor{ansi-cyan-intense}{HTML}{258F8F}
    \definecolor{ansi-white}{HTML}{C5C1B4}
    \definecolor{ansi-white-intense}{HTML}{A1A6B2}

    % commands and environments needed by pandoc snippets
    % extracted from the output of `pandoc -s`
    \providecommand{\tightlist}{%
      \setlength{\itemsep}{0pt}\setlength{\parskip}{0pt}}
    \DefineVerbatimEnvironment{Highlighting}{Verbatim}{commandchars=\\\{\}}
    % Add ',fontsize=\small' for more characters per line
    \newenvironment{Shaded}{}{}
    \newcommand{\KeywordTok}[1]{\textcolor[rgb]{0.00,0.44,0.13}{\textbf{{#1}}}}
    \newcommand{\DataTypeTok}[1]{\textcolor[rgb]{0.56,0.13,0.00}{{#1}}}
    \newcommand{\DecValTok}[1]{\textcolor[rgb]{0.25,0.63,0.44}{{#1}}}
    \newcommand{\BaseNTok}[1]{\textcolor[rgb]{0.25,0.63,0.44}{{#1}}}
    \newcommand{\FloatTok}[1]{\textcolor[rgb]{0.25,0.63,0.44}{{#1}}}
    \newcommand{\CharTok}[1]{\textcolor[rgb]{0.25,0.44,0.63}{{#1}}}
    \newcommand{\StringTok}[1]{\textcolor[rgb]{0.25,0.44,0.63}{{#1}}}
    \newcommand{\CommentTok}[1]{\textcolor[rgb]{0.38,0.63,0.69}{\textit{{#1}}}}
    \newcommand{\OtherTok}[1]{\textcolor[rgb]{0.00,0.44,0.13}{{#1}}}
    \newcommand{\AlertTok}[1]{\textcolor[rgb]{1.00,0.00,0.00}{\textbf{{#1}}}}
    \newcommand{\FunctionTok}[1]{\textcolor[rgb]{0.02,0.16,0.49}{{#1}}}
    \newcommand{\RegionMarkerTok}[1]{{#1}}
    \newcommand{\ErrorTok}[1]{\textcolor[rgb]{1.00,0.00,0.00}{\textbf{{#1}}}}
    \newcommand{\NormalTok}[1]{{#1}}
    
    % Additional commands for more recent versions of Pandoc
    \newcommand{\ConstantTok}[1]{\textcolor[rgb]{0.53,0.00,0.00}{{#1}}}
    \newcommand{\SpecialCharTok}[1]{\textcolor[rgb]{0.25,0.44,0.63}{{#1}}}
    \newcommand{\VerbatimStringTok}[1]{\textcolor[rgb]{0.25,0.44,0.63}{{#1}}}
    \newcommand{\SpecialStringTok}[1]{\textcolor[rgb]{0.73,0.40,0.53}{{#1}}}
    \newcommand{\ImportTok}[1]{{#1}}
    \newcommand{\DocumentationTok}[1]{\textcolor[rgb]{0.73,0.13,0.13}{\textit{{#1}}}}
    \newcommand{\AnnotationTok}[1]{\textcolor[rgb]{0.38,0.63,0.69}{\textbf{\textit{{#1}}}}}
    \newcommand{\CommentVarTok}[1]{\textcolor[rgb]{0.38,0.63,0.69}{\textbf{\textit{{#1}}}}}
    \newcommand{\VariableTok}[1]{\textcolor[rgb]{0.10,0.09,0.49}{{#1}}}
    \newcommand{\ControlFlowTok}[1]{\textcolor[rgb]{0.00,0.44,0.13}{\textbf{{#1}}}}
    \newcommand{\OperatorTok}[1]{\textcolor[rgb]{0.40,0.40,0.40}{{#1}}}
    \newcommand{\BuiltInTok}[1]{{#1}}
    \newcommand{\ExtensionTok}[1]{{#1}}
    \newcommand{\PreprocessorTok}[1]{\textcolor[rgb]{0.74,0.48,0.00}{{#1}}}
    \newcommand{\AttributeTok}[1]{\textcolor[rgb]{0.49,0.56,0.16}{{#1}}}
    \newcommand{\InformationTok}[1]{\textcolor[rgb]{0.38,0.63,0.69}{\textbf{\textit{{#1}}}}}
    \newcommand{\WarningTok}[1]{\textcolor[rgb]{0.38,0.63,0.69}{\textbf{\textit{{#1}}}}}
    
    
    % Define a nice break command that doesn't care if a line doesn't already
    % exist.
    \def\br{\hspace*{\fill} \\* }
    % Math Jax compatability definitions
    \def\gt{>}
    \def\lt{<}
    % Document parameters
    \title{linear\_regression}
    
    
    

    % Pygments definitions
    
\makeatletter
\def\PY@reset{\let\PY@it=\relax \let\PY@bf=\relax%
    \let\PY@ul=\relax \let\PY@tc=\relax%
    \let\PY@bc=\relax \let\PY@ff=\relax}
\def\PY@tok#1{\csname PY@tok@#1\endcsname}
\def\PY@toks#1+{\ifx\relax#1\empty\else%
    \PY@tok{#1}\expandafter\PY@toks\fi}
\def\PY@do#1{\PY@bc{\PY@tc{\PY@ul{%
    \PY@it{\PY@bf{\PY@ff{#1}}}}}}}
\def\PY#1#2{\PY@reset\PY@toks#1+\relax+\PY@do{#2}}

\expandafter\def\csname PY@tok@w\endcsname{\def\PY@tc##1{\textcolor[rgb]{0.73,0.73,0.73}{##1}}}
\expandafter\def\csname PY@tok@c\endcsname{\let\PY@it=\textit\def\PY@tc##1{\textcolor[rgb]{0.25,0.50,0.50}{##1}}}
\expandafter\def\csname PY@tok@cp\endcsname{\def\PY@tc##1{\textcolor[rgb]{0.74,0.48,0.00}{##1}}}
\expandafter\def\csname PY@tok@k\endcsname{\let\PY@bf=\textbf\def\PY@tc##1{\textcolor[rgb]{0.00,0.50,0.00}{##1}}}
\expandafter\def\csname PY@tok@kp\endcsname{\def\PY@tc##1{\textcolor[rgb]{0.00,0.50,0.00}{##1}}}
\expandafter\def\csname PY@tok@kt\endcsname{\def\PY@tc##1{\textcolor[rgb]{0.69,0.00,0.25}{##1}}}
\expandafter\def\csname PY@tok@o\endcsname{\def\PY@tc##1{\textcolor[rgb]{0.40,0.40,0.40}{##1}}}
\expandafter\def\csname PY@tok@ow\endcsname{\let\PY@bf=\textbf\def\PY@tc##1{\textcolor[rgb]{0.67,0.13,1.00}{##1}}}
\expandafter\def\csname PY@tok@nb\endcsname{\def\PY@tc##1{\textcolor[rgb]{0.00,0.50,0.00}{##1}}}
\expandafter\def\csname PY@tok@nf\endcsname{\def\PY@tc##1{\textcolor[rgb]{0.00,0.00,1.00}{##1}}}
\expandafter\def\csname PY@tok@nc\endcsname{\let\PY@bf=\textbf\def\PY@tc##1{\textcolor[rgb]{0.00,0.00,1.00}{##1}}}
\expandafter\def\csname PY@tok@nn\endcsname{\let\PY@bf=\textbf\def\PY@tc##1{\textcolor[rgb]{0.00,0.00,1.00}{##1}}}
\expandafter\def\csname PY@tok@ne\endcsname{\let\PY@bf=\textbf\def\PY@tc##1{\textcolor[rgb]{0.82,0.25,0.23}{##1}}}
\expandafter\def\csname PY@tok@nv\endcsname{\def\PY@tc##1{\textcolor[rgb]{0.10,0.09,0.49}{##1}}}
\expandafter\def\csname PY@tok@no\endcsname{\def\PY@tc##1{\textcolor[rgb]{0.53,0.00,0.00}{##1}}}
\expandafter\def\csname PY@tok@nl\endcsname{\def\PY@tc##1{\textcolor[rgb]{0.63,0.63,0.00}{##1}}}
\expandafter\def\csname PY@tok@ni\endcsname{\let\PY@bf=\textbf\def\PY@tc##1{\textcolor[rgb]{0.60,0.60,0.60}{##1}}}
\expandafter\def\csname PY@tok@na\endcsname{\def\PY@tc##1{\textcolor[rgb]{0.49,0.56,0.16}{##1}}}
\expandafter\def\csname PY@tok@nt\endcsname{\let\PY@bf=\textbf\def\PY@tc##1{\textcolor[rgb]{0.00,0.50,0.00}{##1}}}
\expandafter\def\csname PY@tok@nd\endcsname{\def\PY@tc##1{\textcolor[rgb]{0.67,0.13,1.00}{##1}}}
\expandafter\def\csname PY@tok@s\endcsname{\def\PY@tc##1{\textcolor[rgb]{0.73,0.13,0.13}{##1}}}
\expandafter\def\csname PY@tok@sd\endcsname{\let\PY@it=\textit\def\PY@tc##1{\textcolor[rgb]{0.73,0.13,0.13}{##1}}}
\expandafter\def\csname PY@tok@si\endcsname{\let\PY@bf=\textbf\def\PY@tc##1{\textcolor[rgb]{0.73,0.40,0.53}{##1}}}
\expandafter\def\csname PY@tok@se\endcsname{\let\PY@bf=\textbf\def\PY@tc##1{\textcolor[rgb]{0.73,0.40,0.13}{##1}}}
\expandafter\def\csname PY@tok@sr\endcsname{\def\PY@tc##1{\textcolor[rgb]{0.73,0.40,0.53}{##1}}}
\expandafter\def\csname PY@tok@ss\endcsname{\def\PY@tc##1{\textcolor[rgb]{0.10,0.09,0.49}{##1}}}
\expandafter\def\csname PY@tok@sx\endcsname{\def\PY@tc##1{\textcolor[rgb]{0.00,0.50,0.00}{##1}}}
\expandafter\def\csname PY@tok@m\endcsname{\def\PY@tc##1{\textcolor[rgb]{0.40,0.40,0.40}{##1}}}
\expandafter\def\csname PY@tok@gh\endcsname{\let\PY@bf=\textbf\def\PY@tc##1{\textcolor[rgb]{0.00,0.00,0.50}{##1}}}
\expandafter\def\csname PY@tok@gu\endcsname{\let\PY@bf=\textbf\def\PY@tc##1{\textcolor[rgb]{0.50,0.00,0.50}{##1}}}
\expandafter\def\csname PY@tok@gd\endcsname{\def\PY@tc##1{\textcolor[rgb]{0.63,0.00,0.00}{##1}}}
\expandafter\def\csname PY@tok@gi\endcsname{\def\PY@tc##1{\textcolor[rgb]{0.00,0.63,0.00}{##1}}}
\expandafter\def\csname PY@tok@gr\endcsname{\def\PY@tc##1{\textcolor[rgb]{1.00,0.00,0.00}{##1}}}
\expandafter\def\csname PY@tok@ge\endcsname{\let\PY@it=\textit}
\expandafter\def\csname PY@tok@gs\endcsname{\let\PY@bf=\textbf}
\expandafter\def\csname PY@tok@gp\endcsname{\let\PY@bf=\textbf\def\PY@tc##1{\textcolor[rgb]{0.00,0.00,0.50}{##1}}}
\expandafter\def\csname PY@tok@go\endcsname{\def\PY@tc##1{\textcolor[rgb]{0.53,0.53,0.53}{##1}}}
\expandafter\def\csname PY@tok@gt\endcsname{\def\PY@tc##1{\textcolor[rgb]{0.00,0.27,0.87}{##1}}}
\expandafter\def\csname PY@tok@err\endcsname{\def\PY@bc##1{\setlength{\fboxsep}{0pt}\fcolorbox[rgb]{1.00,0.00,0.00}{1,1,1}{\strut ##1}}}
\expandafter\def\csname PY@tok@kc\endcsname{\let\PY@bf=\textbf\def\PY@tc##1{\textcolor[rgb]{0.00,0.50,0.00}{##1}}}
\expandafter\def\csname PY@tok@kd\endcsname{\let\PY@bf=\textbf\def\PY@tc##1{\textcolor[rgb]{0.00,0.50,0.00}{##1}}}
\expandafter\def\csname PY@tok@kn\endcsname{\let\PY@bf=\textbf\def\PY@tc##1{\textcolor[rgb]{0.00,0.50,0.00}{##1}}}
\expandafter\def\csname PY@tok@kr\endcsname{\let\PY@bf=\textbf\def\PY@tc##1{\textcolor[rgb]{0.00,0.50,0.00}{##1}}}
\expandafter\def\csname PY@tok@bp\endcsname{\def\PY@tc##1{\textcolor[rgb]{0.00,0.50,0.00}{##1}}}
\expandafter\def\csname PY@tok@fm\endcsname{\def\PY@tc##1{\textcolor[rgb]{0.00,0.00,1.00}{##1}}}
\expandafter\def\csname PY@tok@vc\endcsname{\def\PY@tc##1{\textcolor[rgb]{0.10,0.09,0.49}{##1}}}
\expandafter\def\csname PY@tok@vg\endcsname{\def\PY@tc##1{\textcolor[rgb]{0.10,0.09,0.49}{##1}}}
\expandafter\def\csname PY@tok@vi\endcsname{\def\PY@tc##1{\textcolor[rgb]{0.10,0.09,0.49}{##1}}}
\expandafter\def\csname PY@tok@vm\endcsname{\def\PY@tc##1{\textcolor[rgb]{0.10,0.09,0.49}{##1}}}
\expandafter\def\csname PY@tok@sa\endcsname{\def\PY@tc##1{\textcolor[rgb]{0.73,0.13,0.13}{##1}}}
\expandafter\def\csname PY@tok@sb\endcsname{\def\PY@tc##1{\textcolor[rgb]{0.73,0.13,0.13}{##1}}}
\expandafter\def\csname PY@tok@sc\endcsname{\def\PY@tc##1{\textcolor[rgb]{0.73,0.13,0.13}{##1}}}
\expandafter\def\csname PY@tok@dl\endcsname{\def\PY@tc##1{\textcolor[rgb]{0.73,0.13,0.13}{##1}}}
\expandafter\def\csname PY@tok@s2\endcsname{\def\PY@tc##1{\textcolor[rgb]{0.73,0.13,0.13}{##1}}}
\expandafter\def\csname PY@tok@sh\endcsname{\def\PY@tc##1{\textcolor[rgb]{0.73,0.13,0.13}{##1}}}
\expandafter\def\csname PY@tok@s1\endcsname{\def\PY@tc##1{\textcolor[rgb]{0.73,0.13,0.13}{##1}}}
\expandafter\def\csname PY@tok@mb\endcsname{\def\PY@tc##1{\textcolor[rgb]{0.40,0.40,0.40}{##1}}}
\expandafter\def\csname PY@tok@mf\endcsname{\def\PY@tc##1{\textcolor[rgb]{0.40,0.40,0.40}{##1}}}
\expandafter\def\csname PY@tok@mh\endcsname{\def\PY@tc##1{\textcolor[rgb]{0.40,0.40,0.40}{##1}}}
\expandafter\def\csname PY@tok@mi\endcsname{\def\PY@tc##1{\textcolor[rgb]{0.40,0.40,0.40}{##1}}}
\expandafter\def\csname PY@tok@il\endcsname{\def\PY@tc##1{\textcolor[rgb]{0.40,0.40,0.40}{##1}}}
\expandafter\def\csname PY@tok@mo\endcsname{\def\PY@tc##1{\textcolor[rgb]{0.40,0.40,0.40}{##1}}}
\expandafter\def\csname PY@tok@ch\endcsname{\let\PY@it=\textit\def\PY@tc##1{\textcolor[rgb]{0.25,0.50,0.50}{##1}}}
\expandafter\def\csname PY@tok@cm\endcsname{\let\PY@it=\textit\def\PY@tc##1{\textcolor[rgb]{0.25,0.50,0.50}{##1}}}
\expandafter\def\csname PY@tok@cpf\endcsname{\let\PY@it=\textit\def\PY@tc##1{\textcolor[rgb]{0.25,0.50,0.50}{##1}}}
\expandafter\def\csname PY@tok@c1\endcsname{\let\PY@it=\textit\def\PY@tc##1{\textcolor[rgb]{0.25,0.50,0.50}{##1}}}
\expandafter\def\csname PY@tok@cs\endcsname{\let\PY@it=\textit\def\PY@tc##1{\textcolor[rgb]{0.25,0.50,0.50}{##1}}}

\def\PYZbs{\char`\\}
\def\PYZus{\char`\_}
\def\PYZob{\char`\{}
\def\PYZcb{\char`\}}
\def\PYZca{\char`\^}
\def\PYZam{\char`\&}
\def\PYZlt{\char`\<}
\def\PYZgt{\char`\>}
\def\PYZsh{\char`\#}
\def\PYZpc{\char`\%}
\def\PYZdl{\char`\$}
\def\PYZhy{\char`\-}
\def\PYZsq{\char`\'}
\def\PYZdq{\char`\"}
\def\PYZti{\char`\~}
% for compatibility with earlier versions
\def\PYZat{@}
\def\PYZlb{[}
\def\PYZrb{]}
\makeatother


    % Exact colors from NB
    \definecolor{incolor}{rgb}{0.0, 0.0, 0.5}
    \definecolor{outcolor}{rgb}{0.545, 0.0, 0.0}



    
    % Prevent overflowing lines due to hard-to-break entities
    \sloppy 
    % Setup hyperref package
    \hypersetup{
      breaklinks=true,  % so long urls are correctly broken across lines
      colorlinks=true,
      urlcolor=urlcolor,
      linkcolor=linkcolor,
      citecolor=citecolor,
      }
    % Slightly bigger margins than the latex defaults
    
    \geometry{verbose,tmargin=1in,bmargin=1in,lmargin=1in,rmargin=1in}
    
    

    \begin{document}
    
    
    \maketitle
    
    

    
    Regressão Linear e Correlação

    ~

    ~

    ~

    ~

    ~

    Autor: Yuri José Soares de Araújo

    Matrícula: 1520658

    Curso: Engenharia de Computação

    ~

    ~

    ~

    ~

    Fortaleza, 14 de Outubro de 2018

    ~

    ~

    ~

    ~

    Introdução

    Regressão linear é uma técnica bastante utilizada para descobrir se há e
qual a correlação entre duas variáveis, estas chamadas de variável
independente e dependente. O método de regressão linear é baseado no
coeficiente de Pearson:

    \begin{align}
r = \frac{{}\sum_{i=1}^{n} (x_i - \overline{x})(y_i - \overline{y})}
{\sqrt{\sum_{i=1}^{n} (x_i - \overline{x})^2(y_i - \overline{y})^2}} \\
\end{align}

    Que pode ser resumido em:

    \begin{align}
r = \frac{{S_{xy}}}{{\sqrt{S_{xx}.S_{yy}}}}
\end{align}

    Lembrando ainda que a fórmula descrita acima é valida se, e somente se,
\({1 \leq r \leq 1}\), que é a variância descrita pelo coeficiente de
Pearson. Em termos práticos, um \({r}\) mais próximo de \({-1}\)
significa uma maior correlação negativa, já o contrário, quando o
\({r}\) é próximo de \({1}\), existe uma maior correlação positiva.
Quando o \({r}\) tende a zero podemos interpretar que não há correlação
entre as duas variáveis.

    O algoritmo de regressão linear é também um dos algoritmos de predição
utilizados no Machine Learning mais populares devido a sua simplicidade.
Na seção de análise irei realizar uma pequena demonstração do algoritmo
utilizando a biblioteca Scikit Learn para calcular uma predição baseada
em um dataset pré estabelecido.

    ~

    Objetivo Geral

    Por meio deste Notebook desejo utilizar de Computação Interativa para
demonstrar e analisar dados de difícil compreensão e para isto escolhi a
linguagem de programação Python, bastante reconhecida no meio acadêmico
e com ampla aplicação na área de Data Science juntamente com seu
framework de Análise de Dados, o Jupyter Notebook, que veio trazer
algumas funcionalidade já reconhecidas no R Studio para o Python.

    ~

    Análise e Interpretação dos Dados

    A seguir irei fazer uma demonstração dos algoritmo de Regressão Linear e
seu uso em Machine Learning com predição de dados. Como sugerido na
introdução, irei explicar passo a passo cada resultado para que até
iniciantes em Ciência de Dados ou mesmo em Python possam acompahar.

    Ao final de cada seção, irei fazer uma breve revisão para fixar o
conteúdo ensinado e para que possamos passar para o próximo estágio sem
dúvidas.

    ~

    Começando pelo bloco abaixo, onde apenas chamo as bibliotecas
matemáticas que irei utilizar para me ajudar durante o processo de
análise, manipulação e visualização dos dados.

    \begin{Verbatim}[commandchars=\\\{\}]
{\color{incolor}In [{\color{incolor}295}]:} \PY{o}{\PYZpc{}}\PY{k}{matplotlib} inline
          \PY{k+kn}{import} \PY{n+nn}{matplotlib}\PY{n+nn}{.}\PY{n+nn}{pyplot} \PY{k}{as} \PY{n+nn}{plt} \PY{c+c1}{\PYZsh{} invocando o método pyplot para plotar os dados}
          \PY{k+kn}{import} \PY{n+nn}{numpy} \PY{k}{as} \PY{n+nn}{np} \PY{c+c1}{\PYZsh{} biblioteca de manipulação dos dados}
          \PY{k+kn}{import} \PY{n+nn}{pandas} \PY{k}{as} \PY{n+nn}{pd} \PY{c+c1}{\PYZsh{} o mesmo que a de cima, porém mais simples e mais limitada}
          \PY{k+kn}{from} \PY{n+nn}{sklearn} \PY{k}{import} \PY{n}{linear\PYZus{}model} \PY{c+c1}{\PYZsh{} biblioteca que guarda algoritmos de Machine Learning }
          \PY{k+kn}{from} \PY{n+nn}{sklearn}\PY{n+nn}{.}\PY{n+nn}{model\PYZus{}selection} \PY{k}{import} \PY{n}{train\PYZus{}test\PYZus{}split}
          \PY{k+kn}{from} \PY{n+nn}{sklearn}\PY{n+nn}{.}\PY{n+nn}{metrics} \PY{k}{import} \PY{n}{mean\PYZus{}squared\PYZus{}error}\PY{p}{,} \PY{n}{r2\PYZus{}score} \PY{c+c1}{\PYZsh{} funções de métrica para avaliar o meu treinamento}
          \PY{k+kn}{from} \PY{n+nn}{scipy}\PY{n+nn}{.}\PY{n+nn}{stats} \PY{k}{import} \PY{n}{pearsonr} \PY{c+c1}{\PYZsh{} calcula o coeficiente de Pearson}
          \PY{k+kn}{from} \PY{n+nn}{math} \PY{k}{import} \PY{n}{sqrt}
          \PY{k+kn}{from} \PY{n+nn}{sympy} \PY{k}{import} \PY{n}{Matrix}
\end{Verbatim}


    1a Questão

    Agora vou carregar o dataset da planilha para a memória para que possa
ser manipulado e utilizarei as colunas 7 e 8, equivalentes a coluna
``Número de Pessoas na Família'' e a coluna de ``Lixo Gerado''.

    \begin{Verbatim}[commandchars=\\\{\}]
{\color{incolor}In [{\color{incolor}296}]:} \PY{n}{df} \PY{o}{=} \PY{n}{pd}\PY{o}{.}\PY{n}{read\PYZus{}excel}\PY{p}{(}\PY{l+s+s1}{\PYZsq{}}\PY{l+s+s1}{../data/database.xls}\PY{l+s+s1}{\PYZsq{}}\PY{p}{,} \PY{n}{sheet\PYZus{}name}\PY{o}{=}\PY{l+s+s1}{\PYZsq{}}\PY{l+s+s1}{Dados 3}\PY{l+s+s1}{\PYZsq{}}\PY{p}{,} \PY{n}{usecols}\PY{o}{=}\PY{p}{[}\PY{l+m+mi}{7}\PY{p}{,} \PY{l+m+mi}{8}\PY{p}{]}\PY{p}{)}
\end{Verbatim}


    Exibe as primeiras 5 entradas do dataset, lembrando que assim como
muitas linguagens de programação, o Python considera o primeiro índice
de uma lista como \({0}\), então lembre-se de fazer essa subtração
quando estiver olhando para a planilha. Deixei esta função apenas para
validar se os dados e os nomes das colunas estavam inseridos de forma
correta.

    \begin{Verbatim}[commandchars=\\\{\}]
{\color{incolor}In [{\color{incolor}297}]:} \PY{n}{df}\PY{o}{.}\PY{n}{head}\PY{p}{(}\PY{p}{)}
\end{Verbatim}


\begin{Verbatim}[commandchars=\\\{\}]
{\color{outcolor}Out[{\color{outcolor}297}]:}    Nº de Pessoas na Família  Lixo Gerado
          0                         2           64
          1                         1           32
          2                         1           40
          3                         2           73
          4                         2           64
\end{Verbatim}
            
    \begin{Verbatim}[commandchars=\\\{\}]
{\color{incolor}In [{\color{incolor}298}]:} \PY{n}{pessoas\PYZus{}train} \PY{o}{=} \PY{n}{df}\PY{p}{[}\PY{l+s+s1}{\PYZsq{}}\PY{l+s+s1}{Nº de Pessoas na Família}\PY{l+s+s1}{\PYZsq{}}\PY{p}{]}\PY{o}{.}\PY{n}{values}\PY{p}{[}\PY{p}{:}\PY{l+m+mi}{30}\PY{p}{]}\PY{o}{.}\PY{n}{reshape}\PY{p}{(}\PY{o}{\PYZhy{}}\PY{l+m+mi}{1}\PY{p}{,} \PY{l+m+mi}{1}\PY{p}{)}
          \PY{n}{lixo\PYZus{}train} \PY{o}{=} \PY{n}{df}\PY{p}{[}\PY{l+s+s1}{\PYZsq{}}\PY{l+s+s1}{Lixo Gerado}\PY{l+s+s1}{\PYZsq{}}\PY{p}{]}\PY{o}{.}\PY{n}{values}\PY{p}{[}\PY{p}{:}\PY{l+m+mi}{30}\PY{p}{]}\PY{o}{.}\PY{n}{reshape}\PY{p}{(}\PY{o}{\PYZhy{}}\PY{l+m+mi}{1}\PY{p}{,} \PY{l+m+mi}{1}\PY{p}{)}
          \PY{n}{pessoas\PYZus{}test} \PY{o}{=} \PY{n}{df}\PY{p}{[}\PY{l+s+s1}{\PYZsq{}}\PY{l+s+s1}{Nº de Pessoas na Família}\PY{l+s+s1}{\PYZsq{}}\PY{p}{]}\PY{o}{.}\PY{n}{values}\PY{p}{[}\PY{l+m+mi}{30}\PY{p}{:}\PY{p}{]}\PY{o}{.}\PY{n}{reshape}\PY{p}{(}\PY{o}{\PYZhy{}}\PY{l+m+mi}{1}\PY{p}{,} \PY{l+m+mi}{1}\PY{p}{)}
          \PY{n}{lixo\PYZus{}test} \PY{o}{=} \PY{n}{df}\PY{p}{[}\PY{l+s+s1}{\PYZsq{}}\PY{l+s+s1}{Lixo Gerado}\PY{l+s+s1}{\PYZsq{}}\PY{p}{]}\PY{o}{.}\PY{n}{values}\PY{p}{[}\PY{l+m+mi}{30}\PY{p}{:}\PY{p}{]}\PY{o}{.}\PY{n}{reshape}\PY{p}{(}\PY{o}{\PYZhy{}}\PY{l+m+mi}{1}\PY{p}{,} \PY{l+m+mi}{1}\PY{p}{)}
          \PY{n}{regr} \PY{o}{=} \PY{n}{linear\PYZus{}model}\PY{o}{.}\PY{n}{LinearRegression}\PY{p}{(}\PY{p}{)}
          \PY{n}{regr}\PY{o}{.}\PY{n}{fit}\PY{p}{(}\PY{n}{pessoas\PYZus{}train}\PY{p}{,} \PY{n}{lixo\PYZus{}train}\PY{p}{)}
          \PY{n}{lixo\PYZus{}predict} \PY{o}{=} \PY{n}{regr}\PY{o}{.}\PY{n}{predict}\PY{p}{(}\PY{n}{pessoas\PYZus{}test}\PY{p}{)}
\end{Verbatim}


    \begin{Verbatim}[commandchars=\\\{\}]
{\color{incolor}In [{\color{incolor}299}]:} \PY{n}{plt}\PY{o}{.}\PY{n}{scatter}\PY{p}{(}\PY{n}{pessoas\PYZus{}test}\PY{p}{,} \PY{n}{lixo\PYZus{}test}\PY{p}{,} \PY{n}{color}\PY{o}{=}\PY{l+s+s1}{\PYZsq{}}\PY{l+s+s1}{blue}\PY{l+s+s1}{\PYZsq{}}\PY{p}{)}
          \PY{n}{plt}\PY{o}{.}\PY{n}{plot}\PY{p}{(}\PY{n}{pessoas\PYZus{}test}\PY{p}{,} \PY{n}{lixo\PYZus{}predict}\PY{p}{,} \PY{n}{color}\PY{o}{=}\PY{l+s+s1}{\PYZsq{}}\PY{l+s+s1}{red}\PY{l+s+s1}{\PYZsq{}}\PY{p}{,} \PY{n}{linewidth}\PY{o}{=}\PY{l+m+mi}{1}\PY{p}{,} \PY{n}{label}\PY{o}{=}\PY{l+s+s2}{\PYZdq{}}\PY{l+s+s2}{Reta de Regressão}\PY{l+s+s2}{\PYZdq{}}\PY{p}{)}
          \PY{n}{plt}\PY{o}{.}\PY{n}{legend}\PY{p}{(}\PY{n}{bbox\PYZus{}to\PYZus{}anchor}\PY{o}{=}\PY{p}{(}\PY{l+m+mf}{1.05}\PY{p}{,} \PY{l+m+mi}{1}\PY{p}{)}\PY{p}{,} \PY{n}{loc}\PY{o}{=}\PY{l+m+mi}{2}\PY{p}{,} \PY{n}{borderaxespad}\PY{o}{=}\PY{l+m+mf}{0.}\PY{p}{)}
          \PY{n}{plt}\PY{o}{.}\PY{n}{xlabel}\PY{p}{(}\PY{l+s+s2}{\PYZdq{}}\PY{l+s+s2}{Nº de Pessoas na Família}\PY{l+s+s2}{\PYZdq{}}\PY{p}{)}
          \PY{n}{plt}\PY{o}{.}\PY{n}{ylabel}\PY{p}{(}\PY{l+s+s2}{\PYZdq{}}\PY{l+s+s2}{Lixo Gerado}\PY{l+s+s2}{\PYZdq{}}\PY{p}{)}
          \PY{n}{plt}\PY{o}{.}\PY{n}{show}\PY{p}{(}\PY{p}{)}
\end{Verbatim}


    \begin{center}
    \adjustimage{max size={0.9\linewidth}{0.9\paperheight}}{output_41_0.png}
    \end{center}
    { \hspace*{\fill} \\}
    
    \begin{Verbatim}[commandchars=\\\{\}]
{\color{incolor}In [{\color{incolor}300}]:} \PY{n+nb}{print}\PY{p}{(}\PY{n}{f}\PY{l+s+s1}{\PYZsq{}}\PY{l+s+s1}{O coeficiente angular de }\PY{l+s+si}{\PYZob{}df.columns[1]\PYZcb{}}\PY{l+s+s1}{ é }\PY{l+s+s1}{\PYZob{}}\PY{l+s+s1}{str(regr.coef\PYZus{}[0][0])[:5]\PYZcb{}}\PY{l+s+se}{\PYZbs{}}
          \PY{l+s+s1}{ e a interseção da reta é }\PY{l+s+s1}{\PYZob{}}\PY{l+s+s1}{str(regr.intercept\PYZus{}[0])[:5]\PYZcb{}.}\PY{l+s+se}{\PYZbs{}}
          \PY{l+s+s1}{ A equação resultante é:}\PY{l+s+se}{\PYZbs{}n}\PY{l+s+s1}{ y = }\PY{l+s+s1}{\PYZob{}}\PY{l+s+s1}{str(regr.coef\PYZus{}[0][0])[:5]\PYZcb{}x}\PY{l+s+s1}{\PYZob{}}\PY{l+s+s1}{str(regr.intercept\PYZus{}[0])[:5]\PYZcb{}}\PY{l+s+s1}{\PYZsq{}}\PY{p}{)}
\end{Verbatim}


    \begin{Verbatim}[commandchars=\\\{\}]
O coeficiente angular de Lixo Gerado é 36.06 e a interseção da reta é -7.73. A equação resultante é:
 y = 36.06x-7.73

    \end{Verbatim}

    \begin{Verbatim}[commandchars=\\\{\}]
{\color{incolor}In [{\color{incolor}301}]:} \PY{n+nb}{print}\PY{p}{(}\PY{n}{f}\PY{l+s+s2}{\PYZdq{}}\PY{l+s+s2}{O coeficiente de determinação é }\PY{l+s+s2}{\PYZob{}}\PY{l+s+s2}{str(r2\PYZus{}score(lixo\PYZus{}train, lixo\PYZus{}predict))[:5]\PYZcb{}.}\PY{l+s+s2}{\PYZdq{}}\PY{p}{)}
\end{Verbatim}


    \begin{Verbatim}[commandchars=\\\{\}]
O coeficiente de determinação é -1.31.

    \end{Verbatim}

    \begin{Verbatim}[commandchars=\\\{\}]
{\color{incolor}In [{\color{incolor}302}]:} \PY{n+nb}{print}\PY{p}{(}\PY{n}{f}\PY{l+s+s2}{\PYZdq{}}\PY{l+s+s2}{A estimativa do volume de lixo gerado por uma família de 4 pessoas é}\PY{l+s+se}{\PYZbs{}}
          \PY{l+s+s2}{ }\PY{l+s+s2}{\PYZob{}}\PY{l+s+s2}{str(regr.predict([[4]])[0][0])[:5]\PYZcb{}.}\PY{l+s+s2}{\PYZdq{}}\PY{p}{)}
\end{Verbatim}


    \begin{Verbatim}[commandchars=\\\{\}]
A estimativa do volume de lixo gerado por uma família de 4 pessoas é 136.5.

    \end{Verbatim}

    Média:

    \begin{Verbatim}[commandchars=\\\{\}]
{\color{incolor}In [{\color{incolor}303}]:} \PY{n+nb}{print}\PY{p}{(}\PY{n}{f}\PY{l+s+s1}{\PYZsq{}}\PY{l+s+s1}{Nº de Pessoas na Família: }\PY{l+s+s1}{\PYZob{}}\PY{l+s+s1}{df[}\PY{l+s+s1}{\PYZdq{}}\PY{l+s+s1}{Nº de Pessoas na Família}\PY{l+s+s1}{\PYZdq{}}\PY{l+s+s1}{].mean()\PYZcb{}}\PY{l+s+s1}{\PYZsq{}}\PY{p}{)}
          \PY{n+nb}{print}\PY{p}{(}\PY{n}{f}\PY{l+s+s1}{\PYZsq{}}\PY{l+s+s1}{Lixo Gerado: }\PY{l+s+s1}{\PYZob{}}\PY{l+s+s1}{df[}\PY{l+s+s1}{\PYZdq{}}\PY{l+s+s1}{Lixo Gerado}\PY{l+s+s1}{\PYZdq{}}\PY{l+s+s1}{].mean()\PYZcb{}}\PY{l+s+s1}{\PYZsq{}}\PY{p}{)}
\end{Verbatim}


    \begin{Verbatim}[commandchars=\\\{\}]
Nº de Pessoas na Família: 2.05
Lixo Gerado: 66.15

    \end{Verbatim}

    Mediana:

    \begin{Verbatim}[commandchars=\\\{\}]
{\color{incolor}In [{\color{incolor}304}]:} \PY{n+nb}{print}\PY{p}{(}\PY{n}{f}\PY{l+s+s1}{\PYZsq{}}\PY{l+s+s1}{Nº de Pessoas na Família: }\PY{l+s+s1}{\PYZob{}}\PY{l+s+s1}{df[}\PY{l+s+s1}{\PYZdq{}}\PY{l+s+s1}{Nº de Pessoas na Família}\PY{l+s+s1}{\PYZdq{}}\PY{l+s+s1}{].median()\PYZcb{}}\PY{l+s+s1}{\PYZsq{}}\PY{p}{)}
          \PY{n+nb}{print}\PY{p}{(}\PY{n}{f}\PY{l+s+s1}{\PYZsq{}}\PY{l+s+s1}{Lixo Gerado: }\PY{l+s+s1}{\PYZob{}}\PY{l+s+s1}{df[}\PY{l+s+s1}{\PYZdq{}}\PY{l+s+s1}{Lixo Gerado}\PY{l+s+s1}{\PYZdq{}}\PY{l+s+s1}{].median()\PYZcb{}}\PY{l+s+s1}{\PYZsq{}}\PY{p}{)}
\end{Verbatim}


    \begin{Verbatim}[commandchars=\\\{\}]
Nº de Pessoas na Família: 2.0
Lixo Gerado: 64.0

    \end{Verbatim}

    Desvio padrão:

    \begin{Verbatim}[commandchars=\\\{\}]
{\color{incolor}In [{\color{incolor}305}]:} \PY{n+nb}{print}\PY{p}{(}\PY{n}{f}\PY{l+s+s1}{\PYZsq{}}\PY{l+s+s1}{Nº de Pessoas na Família: }\PY{l+s+s1}{\PYZob{}}\PY{l+s+s1}{df[}\PY{l+s+s1}{\PYZdq{}}\PY{l+s+s1}{Nº de Pessoas na Família}\PY{l+s+s1}{\PYZdq{}}\PY{l+s+s1}{].std()\PYZcb{}}\PY{l+s+s1}{\PYZsq{}}\PY{p}{)}
          \PY{n+nb}{print}\PY{p}{(}\PY{n}{f}\PY{l+s+s1}{\PYZsq{}}\PY{l+s+s1}{Lixo Gerado: }\PY{l+s+s1}{\PYZob{}}\PY{l+s+s1}{df[}\PY{l+s+s1}{\PYZdq{}}\PY{l+s+s1}{Lixo Gerado}\PY{l+s+s1}{\PYZdq{}}\PY{l+s+s1}{].std()\PYZcb{}}\PY{l+s+s1}{\PYZsq{}}\PY{p}{)}
\end{Verbatim}


    \begin{Verbatim}[commandchars=\\\{\}]
Nº de Pessoas na Família: 0.9099264265284941
Lixo Gerado: 33.96899134330244

    \end{Verbatim}

    Variância:

    \begin{Verbatim}[commandchars=\\\{\}]
{\color{incolor}In [{\color{incolor}306}]:} \PY{n+nb}{print}\PY{p}{(}\PY{n}{f}\PY{l+s+s1}{\PYZsq{}}\PY{l+s+s1}{Nº de Pessoas na Família: }\PY{l+s+s1}{\PYZob{}}\PY{l+s+s1}{df[}\PY{l+s+s1}{\PYZdq{}}\PY{l+s+s1}{Nº de Pessoas na Família}\PY{l+s+s1}{\PYZdq{}}\PY{l+s+s1}{].var()\PYZcb{}}\PY{l+s+s1}{\PYZsq{}}\PY{p}{)}
          \PY{n+nb}{print}\PY{p}{(}\PY{n}{f}\PY{l+s+s1}{\PYZsq{}}\PY{l+s+s1}{Lixo Gerado: }\PY{l+s+s1}{\PYZob{}}\PY{l+s+s1}{df[}\PY{l+s+s1}{\PYZdq{}}\PY{l+s+s1}{Lixo Gerado}\PY{l+s+s1}{\PYZdq{}}\PY{l+s+s1}{].var()\PYZcb{}}\PY{l+s+s1}{\PYZsq{}}\PY{p}{)}
\end{Verbatim}


    \begin{Verbatim}[commandchars=\\\{\}]
Nº de Pessoas na Família: 0.8279661016949149
Lixo Gerado: 1153.892372881356

    \end{Verbatim}

    Mais informações sobre o dataset:

    \begin{Verbatim}[commandchars=\\\{\}]
{\color{incolor}In [{\color{incolor}307}]:} \PY{n}{df}\PY{o}{.}\PY{n}{info}\PY{p}{(}\PY{p}{)}
\end{Verbatim}


    \begin{Verbatim}[commandchars=\\\{\}]
<class 'pandas.core.frame.DataFrame'>
RangeIndex: 60 entries, 0 to 59
Data columns (total 2 columns):
Nº de Pessoas na Família    60 non-null int64
Lixo Gerado                 60 non-null int64
dtypes: int64(2)
memory usage: 1.0 KB

    \end{Verbatim}

    ~

    2a Questão

    \begin{Verbatim}[commandchars=\\\{\}]
{\color{incolor}In [{\color{incolor}308}]:} \PY{n}{df2} \PY{o}{=} \PY{n}{pd}\PY{o}{.}\PY{n}{read\PYZus{}excel}\PY{p}{(}\PY{l+s+s1}{\PYZsq{}}\PY{l+s+s1}{../data/database.xls}\PY{l+s+s1}{\PYZsq{}}\PY{p}{,} \PY{n}{sheet\PYZus{}name}\PY{o}{=}\PY{l+s+s1}{\PYZsq{}}\PY{l+s+s1}{Dados 8}\PY{l+s+s1}{\PYZsq{}}\PY{p}{,} \PY{n}{usecols}\PY{o}{=}\PY{p}{[}\PY{l+m+mi}{0}\PY{p}{,} \PY{l+m+mi}{1}\PY{p}{]}\PY{p}{)}
\end{Verbatim}


    \begin{Verbatim}[commandchars=\\\{\}]
{\color{incolor}In [{\color{incolor}309}]:} \PY{n}{df2}\PY{o}{.}\PY{n}{head}\PY{p}{(}\PY{p}{)}
\end{Verbatim}


\begin{Verbatim}[commandchars=\\\{\}]
{\color{outcolor}Out[{\color{outcolor}309}]:}       X      Y
          0  22.0  64.03
          1  20.0  62.47
          2  18.0  54.94
          3  16.0  48.84
          4  14.0  43.73
\end{Verbatim}
            
    \begin{Verbatim}[commandchars=\\\{\}]
{\color{incolor}In [{\color{incolor}310}]:} \PY{n}{x\PYZus{}train} \PY{o}{=} \PY{n}{df2}\PY{p}{[}\PY{l+s+s1}{\PYZsq{}}\PY{l+s+s1}{X}\PY{l+s+s1}{\PYZsq{}}\PY{p}{]}\PY{o}{.}\PY{n}{values}\PY{p}{[}\PY{p}{:}\PY{l+m+mi}{12}\PY{p}{]}\PY{o}{.}\PY{n}{reshape}\PY{p}{(}\PY{o}{\PYZhy{}}\PY{l+m+mi}{1}\PY{p}{,} \PY{l+m+mi}{1}\PY{p}{)}
          \PY{n}{y\PYZus{}train} \PY{o}{=} \PY{n}{df2}\PY{p}{[}\PY{l+s+s1}{\PYZsq{}}\PY{l+s+s1}{Y}\PY{l+s+s1}{\PYZsq{}}\PY{p}{]}\PY{o}{.}\PY{n}{values}\PY{p}{[}\PY{p}{:}\PY{l+m+mi}{12}\PY{p}{]}\PY{o}{.}\PY{n}{reshape}\PY{p}{(}\PY{o}{\PYZhy{}}\PY{l+m+mi}{1}\PY{p}{,} \PY{l+m+mi}{1}\PY{p}{)}
          \PY{n}{x\PYZus{}test} \PY{o}{=} \PY{n}{df2}\PY{p}{[}\PY{l+s+s1}{\PYZsq{}}\PY{l+s+s1}{X}\PY{l+s+s1}{\PYZsq{}}\PY{p}{]}\PY{o}{.}\PY{n}{values}\PY{p}{[}\PY{l+m+mi}{12}\PY{p}{:}\PY{p}{]}\PY{o}{.}\PY{n}{reshape}\PY{p}{(}\PY{o}{\PYZhy{}}\PY{l+m+mi}{1}\PY{p}{,} \PY{l+m+mi}{1}\PY{p}{)}
          \PY{n}{y\PYZus{}test} \PY{o}{=} \PY{n}{df2}\PY{p}{[}\PY{l+s+s1}{\PYZsq{}}\PY{l+s+s1}{Y}\PY{l+s+s1}{\PYZsq{}}\PY{p}{]}\PY{o}{.}\PY{n}{values}\PY{p}{[}\PY{l+m+mi}{12}\PY{p}{:}\PY{p}{]}\PY{o}{.}\PY{n}{reshape}\PY{p}{(}\PY{o}{\PYZhy{}}\PY{l+m+mi}{1}\PY{p}{,} \PY{l+m+mi}{1}\PY{p}{)}
          \PY{n}{regr} \PY{o}{=} \PY{n}{linear\PYZus{}model}\PY{o}{.}\PY{n}{LinearRegression}\PY{p}{(}\PY{p}{)}
          \PY{n}{regr}\PY{o}{.}\PY{n}{fit}\PY{p}{(}\PY{n}{x\PYZus{}train}\PY{p}{,} \PY{n}{y\PYZus{}train}\PY{p}{)}
          \PY{n}{y\PYZus{}predict} \PY{o}{=} \PY{n}{regr}\PY{o}{.}\PY{n}{predict}\PY{p}{(}\PY{n}{x\PYZus{}test}\PY{p}{)}
\end{Verbatim}


    \begin{Verbatim}[commandchars=\\\{\}]
{\color{incolor}In [{\color{incolor}311}]:} \PY{n}{plt}\PY{o}{.}\PY{n}{scatter}\PY{p}{(}\PY{n}{x\PYZus{}test}\PY{p}{,} \PY{n}{y\PYZus{}test}\PY{p}{,} \PY{n}{color}\PY{o}{=}\PY{l+s+s1}{\PYZsq{}}\PY{l+s+s1}{blue}\PY{l+s+s1}{\PYZsq{}}\PY{p}{)}
          \PY{n}{plt}\PY{o}{.}\PY{n}{plot}\PY{p}{(}\PY{n}{x\PYZus{}test}\PY{p}{,} \PY{n}{y\PYZus{}predict}\PY{p}{,} \PY{n}{color}\PY{o}{=}\PY{l+s+s1}{\PYZsq{}}\PY{l+s+s1}{red}\PY{l+s+s1}{\PYZsq{}}\PY{p}{,} \PY{n}{linewidth}\PY{o}{=}\PY{l+m+mi}{1}\PY{p}{,} \PY{n}{label}\PY{o}{=}\PY{l+s+s2}{\PYZdq{}}\PY{l+s+s2}{Reta de Regressão}\PY{l+s+s2}{\PYZdq{}}\PY{p}{)}
          \PY{n}{plt}\PY{o}{.}\PY{n}{legend}\PY{p}{(}\PY{n}{bbox\PYZus{}to\PYZus{}anchor}\PY{o}{=}\PY{p}{(}\PY{l+m+mf}{1.05}\PY{p}{,} \PY{l+m+mi}{1}\PY{p}{)}\PY{p}{,} \PY{n}{loc}\PY{o}{=}\PY{l+m+mi}{2}\PY{p}{,} \PY{n}{borderaxespad}\PY{o}{=}\PY{l+m+mf}{0.}\PY{p}{)}
          \PY{n}{plt}\PY{o}{.}\PY{n}{xlabel}\PY{p}{(}\PY{l+s+s2}{\PYZdq{}}\PY{l+s+s2}{Velocidade}\PY{l+s+s2}{\PYZdq{}}\PY{p}{)}
          \PY{n}{plt}\PY{o}{.}\PY{n}{ylabel}\PY{p}{(}\PY{l+s+s2}{\PYZdq{}}\PY{l+s+s2}{Capacidade}\PY{l+s+s2}{\PYZdq{}}\PY{p}{)}
          \PY{n}{plt}\PY{o}{.}\PY{n}{show}\PY{p}{(}\PY{p}{)}
\end{Verbatim}


    \begin{center}
    \adjustimage{max size={0.9\linewidth}{0.9\paperheight}}{output_60_0.png}
    \end{center}
    { \hspace*{\fill} \\}
    
    \begin{Verbatim}[commandchars=\\\{\}]
{\color{incolor}In [{\color{incolor}312}]:} \PY{n+nb}{print}\PY{p}{(}\PY{n}{f}\PY{l+s+s1}{\PYZsq{}}\PY{l+s+s1}{O coeficiente angular de }\PY{l+s+si}{\PYZob{}df2.columns[1]\PYZcb{}}\PY{l+s+s1}{ é }\PY{l+s+s1}{\PYZob{}}\PY{l+s+s1}{str(regr.coef\PYZus{}[0][0])[:5]\PYZcb{}}\PY{l+s+se}{\PYZbs{}}
          \PY{l+s+s1}{ e a interseção da reta é }\PY{l+s+s1}{\PYZob{}}\PY{l+s+s1}{str(regr.intercept\PYZus{}[0])[:5]\PYZcb{}.}\PY{l+s+se}{\PYZbs{}}
          \PY{l+s+s1}{ A equação resultante é:}\PY{l+s+se}{\PYZbs{}n}\PY{l+s+s1}{ y = }\PY{l+s+s1}{\PYZob{}}\PY{l+s+s1}{str(regr.coef\PYZus{}[0][0])[:5]\PYZcb{}x + }\PY{l+s+s1}{\PYZob{}}\PY{l+s+s1}{str(regr.intercept\PYZus{}[0])[:5]\PYZcb{}}\PY{l+s+s1}{\PYZsq{}}\PY{p}{)}
\end{Verbatim}


    \begin{Verbatim}[commandchars=\\\{\}]
O coeficiente angular de Y é 2.766 e a interseção da reta é 4.991. A equação resultante é:
 y = 2.766x + 4.991

    \end{Verbatim}

    \begin{Verbatim}[commandchars=\\\{\}]
{\color{incolor}In [{\color{incolor}313}]:} \PY{n+nb}{print}\PY{p}{(}\PY{n}{f}\PY{l+s+s2}{\PYZdq{}}\PY{l+s+s2}{O coeficiente de determinação é }\PY{l+s+s2}{\PYZob{}}\PY{l+s+s2}{str(r2\PYZus{}score(y\PYZus{}train, y\PYZus{}predict))[:5]\PYZcb{}.}\PY{l+s+s2}{\PYZdq{}}\PY{p}{)}
\end{Verbatim}


    \begin{Verbatim}[commandchars=\\\{\}]
O coeficiente de determinação é 0.222.

    \end{Verbatim}

    \begin{Verbatim}[commandchars=\\\{\}]
{\color{incolor}In [{\color{incolor}314}]:} \PY{n+nb}{print}\PY{p}{(}\PY{n}{f}\PY{l+s+s2}{\PYZdq{}}\PY{l+s+s2}{A estimativa da capacidade da máquina para uma velocidade de 15 rpm é}\PY{l+s+se}{\PYZbs{}}
          \PY{l+s+s2}{ }\PY{l+s+s2}{\PYZob{}}\PY{l+s+s2}{str(regr.predict([[15]])[0][0])[:5]\PYZcb{}.}\PY{l+s+s2}{\PYZdq{}}\PY{p}{)}
\end{Verbatim}


    \begin{Verbatim}[commandchars=\\\{\}]
A estimativa da capacidade da máquina para uma velocidade de 15 rpm é 46.48.

    \end{Verbatim}

    Média:

    \begin{Verbatim}[commandchars=\\\{\}]
{\color{incolor}In [{\color{incolor}315}]:} \PY{n+nb}{print}\PY{p}{(}\PY{n}{f}\PY{l+s+s1}{\PYZsq{}}\PY{l+s+s1}{Velocidade: }\PY{l+s+s1}{\PYZob{}}\PY{l+s+s1}{df2[}\PY{l+s+s1}{\PYZdq{}}\PY{l+s+s1}{X}\PY{l+s+s1}{\PYZdq{}}\PY{l+s+s1}{].mean()\PYZcb{}}\PY{l+s+s1}{\PYZsq{}}\PY{p}{)}
          \PY{n+nb}{print}\PY{p}{(}\PY{n}{f}\PY{l+s+s1}{\PYZsq{}}\PY{l+s+s1}{Capacidade: }\PY{l+s+s1}{\PYZob{}}\PY{l+s+s1}{str(df2[}\PY{l+s+s1}{\PYZdq{}}\PY{l+s+s1}{Y}\PY{l+s+s1}{\PYZdq{}}\PY{l+s+s1}{].mean())[:6]\PYZcb{}}\PY{l+s+s1}{\PYZsq{}}\PY{p}{)}
\end{Verbatim}


    \begin{Verbatim}[commandchars=\\\{\}]
Velocidade: 17.4375
Capacidade: 52.298

    \end{Verbatim}

    Mediana:

    \begin{Verbatim}[commandchars=\\\{\}]
{\color{incolor}In [{\color{incolor}316}]:} \PY{n+nb}{print}\PY{p}{(}\PY{n}{f}\PY{l+s+s1}{\PYZsq{}}\PY{l+s+s1}{Velocidade: }\PY{l+s+s1}{\PYZob{}}\PY{l+s+s1}{df2[}\PY{l+s+s1}{\PYZdq{}}\PY{l+s+s1}{X}\PY{l+s+s1}{\PYZdq{}}\PY{l+s+s1}{].median()\PYZcb{}}\PY{l+s+s1}{\PYZsq{}}\PY{p}{)}
          \PY{n+nb}{print}\PY{p}{(}\PY{n}{f}\PY{l+s+s1}{\PYZsq{}}\PY{l+s+s1}{Capacidade: }\PY{l+s+s1}{\PYZob{}}\PY{l+s+s1}{df2[}\PY{l+s+s1}{\PYZdq{}}\PY{l+s+s1}{Y}\PY{l+s+s1}{\PYZdq{}}\PY{l+s+s1}{].median()\PYZcb{}}\PY{l+s+s1}{\PYZsq{}}\PY{p}{)}
\end{Verbatim}


    \begin{Verbatim}[commandchars=\\\{\}]
Velocidade: 17.5
Capacidade: 52.035

    \end{Verbatim}

    Desvio padrão:

    \begin{Verbatim}[commandchars=\\\{\}]
{\color{incolor}In [{\color{incolor}317}]:} \PY{n+nb}{print}\PY{p}{(}\PY{n}{f}\PY{l+s+s1}{\PYZsq{}}\PY{l+s+s1}{Velocidade: }\PY{l+s+s1}{\PYZob{}}\PY{l+s+s1}{str(df2[}\PY{l+s+s1}{\PYZdq{}}\PY{l+s+s1}{X}\PY{l+s+s1}{\PYZdq{}}\PY{l+s+s1}{].std())[:4]\PYZcb{}}\PY{l+s+s1}{\PYZsq{}}\PY{p}{)}
          \PY{n+nb}{print}\PY{p}{(}\PY{n}{f}\PY{l+s+s1}{\PYZsq{}}\PY{l+s+s1}{Capacidade: }\PY{l+s+s1}{\PYZob{}}\PY{l+s+s1}{str(df2[}\PY{l+s+s1}{\PYZdq{}}\PY{l+s+s1}{Y}\PY{l+s+s1}{\PYZdq{}}\PY{l+s+s1}{].std())[:5]\PYZcb{}}\PY{l+s+s1}{\PYZsq{}}\PY{p}{)}
\end{Verbatim}


    \begin{Verbatim}[commandchars=\\\{\}]
Velocidade: 3.76
Capacidade: 10.14

    \end{Verbatim}

    Variância:

    \begin{Verbatim}[commandchars=\\\{\}]
{\color{incolor}In [{\color{incolor}318}]:} \PY{n+nb}{print}\PY{p}{(}\PY{n}{f}\PY{l+s+s1}{\PYZsq{}}\PY{l+s+s1}{Velocidade: }\PY{l+s+s1}{\PYZob{}}\PY{l+s+s1}{str(df2[}\PY{l+s+s1}{\PYZdq{}}\PY{l+s+s1}{X}\PY{l+s+s1}{\PYZdq{}}\PY{l+s+s1}{].var())[:5]\PYZcb{}}\PY{l+s+s1}{\PYZsq{}}\PY{p}{)}
          \PY{n+nb}{print}\PY{p}{(}\PY{n}{f}\PY{l+s+s1}{\PYZsq{}}\PY{l+s+s1}{Capacidade: }\PY{l+s+s1}{\PYZob{}}\PY{l+s+s1}{str(df2[}\PY{l+s+s1}{\PYZdq{}}\PY{l+s+s1}{Y}\PY{l+s+s1}{\PYZdq{}}\PY{l+s+s1}{].var())[:6]\PYZcb{}}\PY{l+s+s1}{\PYZsq{}}\PY{p}{)}
\end{Verbatim}


    \begin{Verbatim}[commandchars=\\\{\}]
Velocidade: 14.20
Capacidade: 102.91

    \end{Verbatim}

    Mais informações sobre o dataset:

    \begin{Verbatim}[commandchars=\\\{\}]
{\color{incolor}In [{\color{incolor}319}]:} \PY{n}{df2}\PY{o}{.}\PY{n}{info}\PY{p}{(}\PY{p}{)}
\end{Verbatim}


    \begin{Verbatim}[commandchars=\\\{\}]
<class 'pandas.core.frame.DataFrame'>
RangeIndex: 24 entries, 0 to 23
Data columns (total 2 columns):
X    24 non-null float64
Y    24 non-null float64
dtypes: float64(2)
memory usage: 464.0 bytes

    \end{Verbatim}

    ~

    3a Questão

    \begin{Verbatim}[commandchars=\\\{\}]
{\color{incolor}In [{\color{incolor}320}]:} \PY{n}{df} \PY{o}{=} \PY{n}{pd}\PY{o}{.}\PY{n}{read\PYZus{}excel}\PY{p}{(}\PY{l+s+s1}{\PYZsq{}}\PY{l+s+s1}{../data/database.xls}\PY{l+s+s1}{\PYZsq{}}\PY{p}{,} \PY{n}{sheet\PYZus{}name}\PY{o}{=}\PY{l+s+s1}{\PYZsq{}}\PY{l+s+s1}{Dados 3}\PY{l+s+s1}{\PYZsq{}}\PY{p}{,} \PY{n}{usecols}\PY{o}{=}\PY{p}{[}\PY{l+m+mi}{7}\PY{p}{,} \PY{l+m+mi}{8}\PY{p}{]}\PY{p}{)}
          \PY{n}{df2} \PY{o}{=} \PY{n}{pd}\PY{o}{.}\PY{n}{read\PYZus{}excel}\PY{p}{(}\PY{l+s+s1}{\PYZsq{}}\PY{l+s+s1}{../data/database.xls}\PY{l+s+s1}{\PYZsq{}}\PY{p}{,} \PY{n}{sheet\PYZus{}name}\PY{o}{=}\PY{l+s+s1}{\PYZsq{}}\PY{l+s+s1}{Dados 8}\PY{l+s+s1}{\PYZsq{}}\PY{p}{,} \PY{n}{usecols}\PY{o}{=}\PY{p}{[}\PY{l+m+mi}{0}\PY{p}{,} \PY{l+m+mi}{1}\PY{p}{]}\PY{p}{)}
          \PY{k}{for} \PY{n}{i} \PY{o+ow}{in} \PY{n+nb}{range}\PY{p}{(}\PY{l+m+mi}{0}\PY{p}{,} \PY{n+nb}{len}\PY{p}{(}\PY{n}{df}\PY{p}{[}\PY{l+s+s1}{\PYZsq{}}\PY{l+s+s1}{Lixo Gerado}\PY{l+s+s1}{\PYZsq{}}\PY{p}{]}\PY{o}{.}\PY{n}{values}\PY{p}{)}\PY{p}{)}\PY{p}{:}
              \PY{n}{df}\PY{p}{[}\PY{l+s+s1}{\PYZsq{}}\PY{l+s+s1}{Lixo Gerado}\PY{l+s+s1}{\PYZsq{}}\PY{p}{]}\PY{p}{[}\PY{n}{i}\PY{p}{]} \PY{o}{=} \PY{n}{df}\PY{p}{[}\PY{l+s+s1}{\PYZsq{}}\PY{l+s+s1}{Lixo Gerado}\PY{l+s+s1}{\PYZsq{}}\PY{p}{]}\PY{p}{[}\PY{n}{i}\PY{p}{]}\PY{o}{*}\PY{l+m+mf}{1.15} \PY{o}{+} \PY{l+m+mi}{5}
\end{Verbatim}


    \begin{Verbatim}[commandchars=\\\{\}]
{\color{incolor}In [{\color{incolor}321}]:} \PY{n}{df}\PY{o}{.}\PY{n}{head}\PY{p}{(}\PY{p}{)}
\end{Verbatim}


\begin{Verbatim}[commandchars=\\\{\}]
{\color{outcolor}Out[{\color{outcolor}321}]:}    Nº de Pessoas na Família  Lixo Gerado
          0                         2           78
          1                         1           41
          2                         1           51
          3                         2           88
          4                         2           78
\end{Verbatim}
            
    \begin{Verbatim}[commandchars=\\\{\}]
{\color{incolor}In [{\color{incolor}322}]:} \PY{n}{pessoas\PYZus{}train} \PY{o}{=} \PY{n}{df}\PY{p}{[}\PY{l+s+s1}{\PYZsq{}}\PY{l+s+s1}{Nº de Pessoas na Família}\PY{l+s+s1}{\PYZsq{}}\PY{p}{]}\PY{o}{.}\PY{n}{values}\PY{p}{[}\PY{p}{:}\PY{l+m+mi}{30}\PY{p}{]}\PY{o}{.}\PY{n}{reshape}\PY{p}{(}\PY{o}{\PYZhy{}}\PY{l+m+mi}{1}\PY{p}{,} \PY{l+m+mi}{1}\PY{p}{)}
          \PY{n}{lixo\PYZus{}train} \PY{o}{=} \PY{n}{df}\PY{p}{[}\PY{l+s+s1}{\PYZsq{}}\PY{l+s+s1}{Lixo Gerado}\PY{l+s+s1}{\PYZsq{}}\PY{p}{]}\PY{o}{.}\PY{n}{values}\PY{p}{[}\PY{p}{:}\PY{l+m+mi}{30}\PY{p}{]}\PY{o}{.}\PY{n}{reshape}\PY{p}{(}\PY{o}{\PYZhy{}}\PY{l+m+mi}{1}\PY{p}{,} \PY{l+m+mi}{1}\PY{p}{)}
          \PY{n}{pessoas\PYZus{}test} \PY{o}{=} \PY{n}{df}\PY{p}{[}\PY{l+s+s1}{\PYZsq{}}\PY{l+s+s1}{Nº de Pessoas na Família}\PY{l+s+s1}{\PYZsq{}}\PY{p}{]}\PY{o}{.}\PY{n}{values}\PY{p}{[}\PY{l+m+mi}{30}\PY{p}{:}\PY{p}{]}\PY{o}{.}\PY{n}{reshape}\PY{p}{(}\PY{o}{\PYZhy{}}\PY{l+m+mi}{1}\PY{p}{,} \PY{l+m+mi}{1}\PY{p}{)}
          \PY{n}{lixo\PYZus{}test} \PY{o}{=} \PY{n}{df}\PY{p}{[}\PY{l+s+s1}{\PYZsq{}}\PY{l+s+s1}{Lixo Gerado}\PY{l+s+s1}{\PYZsq{}}\PY{p}{]}\PY{o}{.}\PY{n}{values}\PY{p}{[}\PY{l+m+mi}{30}\PY{p}{:}\PY{p}{]}\PY{o}{.}\PY{n}{reshape}\PY{p}{(}\PY{o}{\PYZhy{}}\PY{l+m+mi}{1}\PY{p}{,} \PY{l+m+mi}{1}\PY{p}{)}
          \PY{n}{regr} \PY{o}{=} \PY{n}{linear\PYZus{}model}\PY{o}{.}\PY{n}{LinearRegression}\PY{p}{(}\PY{p}{)}
          \PY{n}{regr}\PY{o}{.}\PY{n}{fit}\PY{p}{(}\PY{n}{pessoas\PYZus{}train}\PY{p}{,} \PY{n}{lixo\PYZus{}train}\PY{p}{)}
          \PY{n}{lixo\PYZus{}predict} \PY{o}{=} \PY{n}{regr}\PY{o}{.}\PY{n}{predict}\PY{p}{(}\PY{n}{pessoas\PYZus{}test}\PY{p}{)}
\end{Verbatim}


    \begin{Verbatim}[commandchars=\\\{\}]
{\color{incolor}In [{\color{incolor}323}]:} \PY{n}{plt}\PY{o}{.}\PY{n}{scatter}\PY{p}{(}\PY{n}{pessoas\PYZus{}test}\PY{p}{,} \PY{n}{lixo\PYZus{}test}\PY{p}{,} \PY{n}{color}\PY{o}{=}\PY{l+s+s1}{\PYZsq{}}\PY{l+s+s1}{blue}\PY{l+s+s1}{\PYZsq{}}\PY{p}{)}
          \PY{n}{plt}\PY{o}{.}\PY{n}{plot}\PY{p}{(}\PY{n}{pessoas\PYZus{}test}\PY{p}{,} \PY{n}{lixo\PYZus{}predict}\PY{p}{,} \PY{n}{color}\PY{o}{=}\PY{l+s+s1}{\PYZsq{}}\PY{l+s+s1}{red}\PY{l+s+s1}{\PYZsq{}}\PY{p}{,} \PY{n}{linewidth}\PY{o}{=}\PY{l+m+mi}{1}\PY{p}{,} \PY{n}{label}\PY{o}{=}\PY{l+s+s2}{\PYZdq{}}\PY{l+s+s2}{Reta de Regressão}\PY{l+s+s2}{\PYZdq{}}\PY{p}{)}
          \PY{n}{plt}\PY{o}{.}\PY{n}{legend}\PY{p}{(}\PY{n}{bbox\PYZus{}to\PYZus{}anchor}\PY{o}{=}\PY{p}{(}\PY{l+m+mf}{1.05}\PY{p}{,} \PY{l+m+mi}{1}\PY{p}{)}\PY{p}{,} \PY{n}{loc}\PY{o}{=}\PY{l+m+mi}{2}\PY{p}{,} \PY{n}{borderaxespad}\PY{o}{=}\PY{l+m+mf}{0.}\PY{p}{)}
          \PY{n}{plt}\PY{o}{.}\PY{n}{xlabel}\PY{p}{(}\PY{l+s+s2}{\PYZdq{}}\PY{l+s+s2}{Nº de Pessoas na Família}\PY{l+s+s2}{\PYZdq{}}\PY{p}{)}
          \PY{n}{plt}\PY{o}{.}\PY{n}{ylabel}\PY{p}{(}\PY{l+s+s2}{\PYZdq{}}\PY{l+s+s2}{Lixo Gerado}\PY{l+s+s2}{\PYZdq{}}\PY{p}{)}
          \PY{n}{plt}\PY{o}{.}\PY{n}{show}\PY{p}{(}\PY{p}{)}
\end{Verbatim}


    \begin{center}
    \adjustimage{max size={0.9\linewidth}{0.9\paperheight}}{output_79_0.png}
    \end{center}
    { \hspace*{\fill} \\}
    
    \begin{Verbatim}[commandchars=\\\{\}]
{\color{incolor}In [{\color{incolor}324}]:} \PY{n+nb}{print}\PY{p}{(}\PY{n}{f}\PY{l+s+s1}{\PYZsq{}}\PY{l+s+s1}{O coeficiente angular de }\PY{l+s+si}{\PYZob{}df.columns[1]\PYZcb{}}\PY{l+s+s1}{ é }\PY{l+s+s1}{\PYZob{}}\PY{l+s+s1}{str(regr.coef\PYZus{}[0][0])[:5]\PYZcb{}}\PY{l+s+se}{\PYZbs{}}
          \PY{l+s+s1}{ e a interseção da reta é }\PY{l+s+s1}{\PYZob{}}\PY{l+s+s1}{str(regr.intercept\PYZus{}[0])[:5]\PYZcb{}.}\PY{l+s+se}{\PYZbs{}}
          \PY{l+s+s1}{ A equação resultante é:}\PY{l+s+se}{\PYZbs{}n}\PY{l+s+s1}{ y = }\PY{l+s+s1}{\PYZob{}}\PY{l+s+s1}{str(regr.coef\PYZus{}[0][0])[:5]\PYZcb{}x}\PY{l+s+s1}{\PYZob{}}\PY{l+s+s1}{str(regr.intercept\PYZus{}[0])[:5]\PYZcb{}}\PY{l+s+s1}{\PYZsq{}}\PY{p}{)}
\end{Verbatim}


    \begin{Verbatim}[commandchars=\\\{\}]
O coeficiente angular de Lixo Gerado é 41.49 e a interseção da reta é -4.48. A equação resultante é:
 y = 41.49x-4.48

    \end{Verbatim}

    \begin{Verbatim}[commandchars=\\\{\}]
{\color{incolor}In [{\color{incolor}325}]:} \PY{n+nb}{print}\PY{p}{(}\PY{n}{f}\PY{l+s+s2}{\PYZdq{}}\PY{l+s+s2}{O coeficiente de determinação é }\PY{l+s+s2}{\PYZob{}}\PY{l+s+s2}{str(r2\PYZus{}score(lixo\PYZus{}train, lixo\PYZus{}predict))[:5]\PYZcb{}.}\PY{l+s+s2}{\PYZdq{}}\PY{p}{)}
\end{Verbatim}


    \begin{Verbatim}[commandchars=\\\{\}]
O coeficiente de determinação é -1.31.

    \end{Verbatim}

    \begin{Verbatim}[commandchars=\\\{\}]
{\color{incolor}In [{\color{incolor}326}]:} \PY{n+nb}{print}\PY{p}{(}\PY{n}{f}\PY{l+s+s2}{\PYZdq{}}\PY{l+s+s2}{A estimativa do volume de lixo gerado por uma família de 4 pessoas é}\PY{l+s+se}{\PYZbs{}}
          \PY{l+s+s2}{ }\PY{l+s+s2}{\PYZob{}}\PY{l+s+s2}{str(regr.predict([[4]])[0][0])[:5]\PYZcb{}.}\PY{l+s+s2}{\PYZdq{}}\PY{p}{)}
\end{Verbatim}


    \begin{Verbatim}[commandchars=\\\{\}]
A estimativa do volume de lixo gerado por uma família de 4 pessoas é 161.5.

    \end{Verbatim}

    Média:

    \begin{Verbatim}[commandchars=\\\{\}]
{\color{incolor}In [{\color{incolor}327}]:} \PY{n+nb}{print}\PY{p}{(}\PY{n}{f}\PY{l+s+s1}{\PYZsq{}}\PY{l+s+s1}{Nº de Pessoas na Família: }\PY{l+s+s1}{\PYZob{}}\PY{l+s+s1}{df[}\PY{l+s+s1}{\PYZdq{}}\PY{l+s+s1}{Nº de Pessoas na Família}\PY{l+s+s1}{\PYZdq{}}\PY{l+s+s1}{].mean()\PYZcb{}}\PY{l+s+s1}{\PYZsq{}}\PY{p}{)}
          \PY{n+nb}{print}\PY{p}{(}\PY{n}{f}\PY{l+s+s1}{\PYZsq{}}\PY{l+s+s1}{Lixo Gerado: }\PY{l+s+s1}{\PYZob{}}\PY{l+s+s1}{str(df[}\PY{l+s+s1}{\PYZdq{}}\PY{l+s+s1}{Lixo Gerado}\PY{l+s+s1}{\PYZdq{}}\PY{l+s+s1}{].mean())[:5]\PYZcb{}}\PY{l+s+s1}{\PYZsq{}}\PY{p}{)}
\end{Verbatim}


    \begin{Verbatim}[commandchars=\\\{\}]
Nº de Pessoas na Família: 2.05
Lixo Gerado: 80.53

    \end{Verbatim}

    Mediana:

    \begin{Verbatim}[commandchars=\\\{\}]
{\color{incolor}In [{\color{incolor}328}]:} \PY{n+nb}{print}\PY{p}{(}\PY{n}{f}\PY{l+s+s1}{\PYZsq{}}\PY{l+s+s1}{Nº de Pessoas na Família: }\PY{l+s+s1}{\PYZob{}}\PY{l+s+s1}{df[}\PY{l+s+s1}{\PYZdq{}}\PY{l+s+s1}{Nº de Pessoas na Família}\PY{l+s+s1}{\PYZdq{}}\PY{l+s+s1}{].median()\PYZcb{}}\PY{l+s+s1}{\PYZsq{}}\PY{p}{)}
          \PY{n+nb}{print}\PY{p}{(}\PY{n}{f}\PY{l+s+s1}{\PYZsq{}}\PY{l+s+s1}{Lixo Gerado: }\PY{l+s+s1}{\PYZob{}}\PY{l+s+s1}{df[}\PY{l+s+s1}{\PYZdq{}}\PY{l+s+s1}{Lixo Gerado}\PY{l+s+s1}{\PYZdq{}}\PY{l+s+s1}{].median()\PYZcb{}}\PY{l+s+s1}{\PYZsq{}}\PY{p}{)}
\end{Verbatim}


    \begin{Verbatim}[commandchars=\\\{\}]
Nº de Pessoas na Família: 2.0
Lixo Gerado: 78.0

    \end{Verbatim}

    Desvio padrão:

    \begin{Verbatim}[commandchars=\\\{\}]
{\color{incolor}In [{\color{incolor}329}]:} \PY{n+nb}{print}\PY{p}{(}\PY{n}{f}\PY{l+s+s1}{\PYZsq{}}\PY{l+s+s1}{Nº de Pessoas na Família: }\PY{l+s+s1}{\PYZob{}}\PY{l+s+s1}{str(df[}\PY{l+s+s1}{\PYZdq{}}\PY{l+s+s1}{Nº de Pessoas na Família}\PY{l+s+s1}{\PYZdq{}}\PY{l+s+s1}{].std())[:5]\PYZcb{}}\PY{l+s+s1}{\PYZsq{}}\PY{p}{)}
          \PY{n+nb}{print}\PY{p}{(}\PY{n}{f}\PY{l+s+s1}{\PYZsq{}}\PY{l+s+s1}{Lixo Gerado: }\PY{l+s+s1}{\PYZob{}}\PY{l+s+s1}{str(df[}\PY{l+s+s1}{\PYZdq{}}\PY{l+s+s1}{Lixo Gerado}\PY{l+s+s1}{\PYZdq{}}\PY{l+s+s1}{].std())[:5]\PYZcb{}}\PY{l+s+s1}{\PYZsq{}}\PY{p}{)}
\end{Verbatim}


    \begin{Verbatim}[commandchars=\\\{\}]
Nº de Pessoas na Família: 0.909
Lixo Gerado: 39.11

    \end{Verbatim}

    Variância:

    \begin{Verbatim}[commandchars=\\\{\}]
{\color{incolor}In [{\color{incolor}330}]:} \PY{n+nb}{print}\PY{p}{(}\PY{n}{f}\PY{l+s+s1}{\PYZsq{}}\PY{l+s+s1}{Nº de Pessoas na Família: }\PY{l+s+s1}{\PYZob{}}\PY{l+s+s1}{str(df[}\PY{l+s+s1}{\PYZdq{}}\PY{l+s+s1}{Nº de Pessoas na Família}\PY{l+s+s1}{\PYZdq{}}\PY{l+s+s1}{].var())[:5]\PYZcb{}}\PY{l+s+s1}{\PYZsq{}}\PY{p}{)}
          \PY{n+nb}{print}\PY{p}{(}\PY{n}{f}\PY{l+s+s1}{\PYZsq{}}\PY{l+s+s1}{Lixo Gerado: }\PY{l+s+s1}{\PYZob{}}\PY{l+s+s1}{str(df[}\PY{l+s+s1}{\PYZdq{}}\PY{l+s+s1}{Lixo Gerado}\PY{l+s+s1}{\PYZdq{}}\PY{l+s+s1}{].var())[:7]\PYZcb{}}\PY{l+s+s1}{\PYZsq{}}\PY{p}{)}
\end{Verbatim}


    \begin{Verbatim}[commandchars=\\\{\}]
Nº de Pessoas na Família: 0.827
Lixo Gerado: 1529.77

    \end{Verbatim}

    Mais informações sobre o dataset:

    \begin{Verbatim}[commandchars=\\\{\}]
{\color{incolor}In [{\color{incolor}331}]:} \PY{n}{df}\PY{o}{.}\PY{n}{info}\PY{p}{(}\PY{p}{)}
\end{Verbatim}


    \begin{Verbatim}[commandchars=\\\{\}]
<class 'pandas.core.frame.DataFrame'>
RangeIndex: 60 entries, 0 to 59
Data columns (total 2 columns):
Nº de Pessoas na Família    60 non-null int64
Lixo Gerado                 60 non-null int64
dtypes: int64(2)
memory usage: 1.0 KB

    \end{Verbatim}

    ~

    4a Questão

    \begin{Verbatim}[commandchars=\\\{\}]
{\color{incolor}In [{\color{incolor}332}]:} \PY{n}{x} \PY{o}{=} \PY{p}{[}\PY{l+m+mi}{4}\PY{p}{,} \PY{l+m+mi}{5}\PY{p}{,} \PY{l+m+mi}{4}\PY{p}{,} \PY{l+m+mi}{5}\PY{p}{,} \PY{l+m+mi}{8}\PY{p}{,} \PY{l+m+mi}{9}\PY{p}{,} \PY{l+m+mi}{10}\PY{p}{,} \PY{l+m+mi}{11}\PY{p}{,} \PY{l+m+mi}{12}\PY{p}{,} \PY{l+m+mi}{12}\PY{p}{]}
          \PY{n}{y} \PY{o}{=} \PY{p}{[}\PY{l+m+mi}{1}\PY{p}{,} \PY{l+m+mi}{1}\PY{p}{,} \PY{l+m+mi}{2}\PY{p}{,} \PY{l+m+mi}{3}\PY{p}{,} \PY{l+m+mi}{3}\PY{p}{,} \PY{l+m+mi}{5}\PY{p}{,} \PY{l+m+mi}{5}\PY{p}{,} \PY{l+m+mi}{6}\PY{p}{,} \PY{l+m+mi}{6}\PY{p}{,} \PY{l+m+mi}{6}\PY{p}{]}
          \PY{n}{z} \PY{o}{=} \PY{p}{[}\PY{l+m+mi}{6}\PY{p}{,} \PY{l+m+mi}{7}\PY{p}{,} \PY{l+m+mi}{10}\PY{p}{,} \PY{l+m+mi}{10}\PY{p}{,} \PY{l+m+mi}{11}\PY{p}{,} \PY{l+m+mi}{9}\PY{p}{,} \PY{l+m+mi}{12}\PY{p}{,} \PY{l+m+mi}{10}\PY{p}{,} \PY{l+m+mi}{11}\PY{p}{,} \PY{l+m+mi}{14}\PY{p}{]}
          
          \PY{n}{df} \PY{o}{=} \PY{n}{pd}\PY{o}{.}\PY{n}{DataFrame}\PY{p}{(}\PY{p}{\PYZob{}}\PY{l+s+s1}{\PYZsq{}}\PY{l+s+s1}{x}\PY{l+s+s1}{\PYZsq{}}\PY{p}{:} \PY{n}{x}\PY{p}{,} \PY{l+s+s1}{\PYZsq{}}\PY{l+s+s1}{y}\PY{l+s+s1}{\PYZsq{}}\PY{p}{:} \PY{n}{y}\PY{p}{,} \PY{l+s+s1}{\PYZsq{}}\PY{l+s+s1}{z}\PY{l+s+s1}{\PYZsq{}}\PY{p}{:} \PY{n}{z}\PY{p}{\PYZcb{}}\PY{p}{)}
\end{Verbatim}


    \begin{Verbatim}[commandchars=\\\{\}]
{\color{incolor}In [{\color{incolor}333}]:} \PY{n}{df}\PY{o}{.}\PY{n}{head}\PY{p}{(}\PY{p}{)}
\end{Verbatim}


\begin{Verbatim}[commandchars=\\\{\}]
{\color{outcolor}Out[{\color{outcolor}333}]:}    x  y   z
          0  4  1   6
          1  5  1   7
          2  4  2  10
          3  5  3  10
          4  8  3  11
\end{Verbatim}
            
    \begin{Verbatim}[commandchars=\\\{\}]
{\color{incolor}In [{\color{incolor}334}]:} \PY{n+nb}{print}\PY{p}{(}\PY{n}{f}\PY{l+s+s1}{\PYZsq{}}\PY{l+s+s1}{O coeficiente de correlação de x pra y é }\PY{l+s+s1}{\PYZob{}}\PY{l+s+s1}{str(pearsonr(x, y))[1:5]\PYZcb{} e o de x para z é}\PY{l+s+se}{\PYZbs{}}
          \PY{l+s+s1}{ }\PY{l+s+s1}{\PYZob{}}\PY{l+s+s1}{str(pearsonr(x, z))[1:5]\PYZcb{}, portanto Y é a melhor variável para estimar o valor de X}\PY{l+s+s1}{\PYZsq{}}\PY{p}{)}
\end{Verbatim}


    \begin{Verbatim}[commandchars=\\\{\}]
O coeficiente de correlação de x pra y é 0.94 e o de x para z é 0.70, portanto Y é a melhor variável para estimar o valor de X

    \end{Verbatim}

    ~

    5a questão

    \begin{Verbatim}[commandchars=\\\{\}]
{\color{incolor}In [{\color{incolor}344}]:} \PY{n}{df} \PY{o}{=} \PY{n}{pd}\PY{o}{.}\PY{n}{DataFrame}\PY{p}{(}\PY{p}{\PYZob{}}\PY{l+s+s1}{\PYZsq{}}\PY{l+s+s1}{Estudante}\PY{l+s+s1}{\PYZsq{}}\PY{p}{:} \PY{p}{[}\PY{l+m+mi}{1}\PY{p}{,} \PY{l+m+mi}{2}\PY{p}{,} \PY{l+m+mi}{3}\PY{p}{,} \PY{l+m+mi}{4}\PY{p}{,} \PY{l+m+mi}{5}\PY{p}{,} \PY{l+m+mi}{6}\PY{p}{,} \PY{l+m+mi}{7}\PY{p}{,} \PY{l+m+mi}{8}\PY{p}{,} \PY{l+m+mi}{9}\PY{p}{]}\PY{p}{,}\PYZbs{}
                             \PY{l+s+s1}{\PYZsq{}}\PY{l+s+s1}{Horas Diárias de Estudo}\PY{l+s+s1}{\PYZsq{}}\PY{p}{:} \PY{p}{[}\PY{l+m+mi}{1}\PY{p}{,} \PY{l+m+mf}{1.5}\PY{p}{,} \PY{l+m+mf}{2.5}\PY{p}{,} \PY{l+m+mf}{2.5}\PY{p}{,} \PY{l+m+mi}{3}\PY{p}{,} \PY{l+m+mf}{3.5}\PY{p}{,} \PY{l+m+mi}{4}\PY{p}{,} \PY{l+m+mf}{4.5}\PY{p}{,} \PY{l+m+mi}{5}\PY{p}{]}\PY{p}{,}\PYZbs{}
                             \PY{l+s+s1}{\PYZsq{}}\PY{l+s+s1}{Desempenho na Disciplina}\PY{l+s+s1}{\PYZsq{}}\PY{p}{:} \PY{p}{[}\PY{l+m+mi}{60}\PY{p}{,} \PY{l+m+mi}{51}\PY{p}{,} \PY{l+m+mi}{70}\PY{p}{,} \PY{l+m+mi}{80}\PY{p}{,} \PY{l+m+mi}{86}\PY{p}{,} \PY{l+m+mi}{50}\PY{p}{,} \PY{l+m+mi}{93}\PY{p}{,} \PY{l+m+mi}{94}\PY{p}{,} \PY{l+m+mi}{88}\PY{p}{]}\PY{p}{\PYZcb{}}\PY{p}{)}
\end{Verbatim}


    \begin{Verbatim}[commandchars=\\\{\}]
{\color{incolor}In [{\color{incolor}345}]:} \PY{n}{df}\PY{o}{.}\PY{n}{head}\PY{p}{(}\PY{p}{)}
\end{Verbatim}


\begin{Verbatim}[commandchars=\\\{\}]
{\color{outcolor}Out[{\color{outcolor}345}]:}    Estudante  Horas Diárias de Estudo  Desempenho na Disciplina
          0          1                      1.0                        60
          1          2                      1.5                        51
          2          3                      2.5                        70
          3          4                      2.5                        80
          4          5                      3.0                        86
\end{Verbatim}
            
    \begin{Verbatim}[commandchars=\\\{\}]
{\color{incolor}In [{\color{incolor}346}]:} \PY{n}{reduced\PYZus{}form}\PY{p}{,} \PY{n}{inds} \PY{o}{=} \PY{n}{Matrix}\PY{p}{(}\PY{n}{df}\PY{p}{[}\PY{p}{[}\PY{l+s+s1}{\PYZsq{}}\PY{l+s+s1}{Desempenho na Disciplina}\PY{l+s+s1}{\PYZsq{}}\PY{p}{,} \PY{l+s+s1}{\PYZsq{}}\PY{l+s+s1}{Horas Diárias de Estudo}\PY{l+s+s1}{\PYZsq{}}\PY{p}{]}\PY{p}{]}\PY{o}{.}\PY{n}{values}\PY{p}{)}\PY{o}{.}\PY{n}{rref}\PY{p}{(}\PY{p}{)}
\end{Verbatim}


    \begin{Verbatim}[commandchars=\\\{\}]
{\color{incolor}In [{\color{incolor}347}]:} \PY{n}{reduced\PYZus{}form}
\end{Verbatim}


\begin{Verbatim}[commandchars=\\\{\}]
{\color{outcolor}Out[{\color{outcolor}347}]:} Matrix([
          [1, 0],
          [0, 1],
          [0, 0],
          [0, 0],
          [0, 0],
          [0, 0],
          [0, 0],
          [0, 0],
          [0, 0]])
\end{Verbatim}
            
    \begin{Verbatim}[commandchars=\\\{\}]
{\color{incolor}In [{\color{incolor}348}]:} \PY{n}{inds}
\end{Verbatim}


\begin{Verbatim}[commandchars=\\\{\}]
{\color{outcolor}Out[{\color{outcolor}348}]:} (0, 1)
\end{Verbatim}
            
    O método \({rref}\) me retorna os índices das colunas linearmente
independentes, sendo a coluna 0 equivalente a `Desempenho na Disciplina'
e a coluna 1 equivalente a `Horas Diárias de Estudo'.

    \begin{Verbatim}[commandchars=\\\{\}]
{\color{incolor}In [{\color{incolor}359}]:} \PY{n}{x} \PY{o}{=} \PY{n}{df}\PY{p}{[}\PY{l+s+s1}{\PYZsq{}}\PY{l+s+s1}{Horas Diárias de Estudo}\PY{l+s+s1}{\PYZsq{}}\PY{p}{]}\PY{o}{.}\PY{n}{values}
          \PY{n}{y} \PY{o}{=} \PY{n}{df}\PY{p}{[}\PY{l+s+s1}{\PYZsq{}}\PY{l+s+s1}{Desempenho na Disciplina}\PY{l+s+s1}{\PYZsq{}}\PY{p}{]}\PY{o}{.}\PY{n}{values}
          \PY{n}{plt}\PY{o}{.}\PY{n}{scatter}\PY{p}{(}\PY{n}{x}\PY{p}{,} \PY{n}{y}\PY{p}{,} \PY{n}{color}\PY{o}{=}\PY{l+s+s1}{\PYZsq{}}\PY{l+s+s1}{blue}\PY{l+s+s1}{\PYZsq{}}\PY{p}{)}
          \PY{n}{plt}\PY{o}{.}\PY{n}{xlabel}\PY{p}{(}\PY{l+s+s2}{\PYZdq{}}\PY{l+s+s2}{Horas Diárias de Estudo}\PY{l+s+s2}{\PYZdq{}}\PY{p}{)}
          \PY{n}{plt}\PY{o}{.}\PY{n}{ylabel}\PY{p}{(}\PY{l+s+s2}{\PYZdq{}}\PY{l+s+s2}{Desempenho na Disciplina}\PY{l+s+s2}{\PYZdq{}}\PY{p}{)}
          \PY{n}{plt}\PY{o}{.}\PY{n}{show}\PY{p}{(}\PY{p}{)}
\end{Verbatim}


    \begin{center}
    \adjustimage{max size={0.9\linewidth}{0.9\paperheight}}{output_106_0.png}
    \end{center}
    { \hspace*{\fill} \\}
    
    \begin{Verbatim}[commandchars=\\\{\}]
{\color{incolor}In [{\color{incolor}360}]:} \PY{n+nb}{print}\PY{p}{(}\PY{n}{f}\PY{l+s+s1}{\PYZsq{}}\PY{l+s+s1}{O coeficiente de correlação de x pra y é }\PY{l+s+s1}{\PYZob{}}\PY{l+s+s1}{str(pearsonr(x, y))[1:5]\PYZcb{}.}\PY{l+s+s1}{\PYZsq{}}\PY{p}{)}
\end{Verbatim}


    \begin{Verbatim}[commandchars=\\\{\}]
O coeficiente de correlação de x pra y é 0.67.

    \end{Verbatim}

    \begin{Verbatim}[commandchars=\\\{\}]
{\color{incolor}In [{\color{incolor}361}]:} \PY{n}{regr} \PY{o}{=} \PY{n}{linear\PYZus{}model}\PY{o}{.}\PY{n}{LinearRegression}\PY{p}{(}\PY{p}{)}
          \PY{n}{regr}\PY{o}{.}\PY{n}{fit}\PY{p}{(}\PY{n}{x}\PY{o}{.}\PY{n}{reshape}\PY{p}{(}\PY{o}{\PYZhy{}}\PY{l+m+mi}{1}\PY{p}{,} \PY{l+m+mi}{1}\PY{p}{)}\PY{p}{,} \PY{n}{y}\PY{o}{.}\PY{n}{reshape}\PY{p}{(}\PY{o}{\PYZhy{}}\PY{l+m+mi}{1}\PY{p}{,} \PY{l+m+mi}{1}\PY{p}{)}\PY{p}{)}
          \PY{n}{y\PYZus{}predict} \PY{o}{=} \PY{n}{regr}\PY{o}{.}\PY{n}{predict}\PY{p}{(}\PY{p}{[}\PY{p}{[}\PY{l+m+mi}{3}\PY{p}{]}\PY{p}{]}\PY{p}{)}
\end{Verbatim}


    \begin{Verbatim}[commandchars=\\\{\}]
{\color{incolor}In [{\color{incolor}366}]:} \PY{n+nb}{print}\PY{p}{(}\PY{n}{f}\PY{l+s+s1}{\PYZsq{}}\PY{l+s+s1}{Para 3 horas de estudo é estimado um desempenho equivalente a }\PY{l+s+s1}{\PYZob{}}\PY{l+s+s1}{str(y\PYZus{}predict[0][0])[:5]\PYZcb{}.}\PY{l+s+s1}{\PYZsq{}}\PY{p}{)}
\end{Verbatim}


    \begin{Verbatim}[commandchars=\\\{\}]
Para 3 horas de estudo é estimado um desempenho equivalente a 74.17.

    \end{Verbatim}

    Média:

    \begin{Verbatim}[commandchars=\\\{\}]
{\color{incolor}In [{\color{incolor}369}]:} \PY{n+nb}{print}\PY{p}{(}\PY{n}{f}\PY{l+s+s1}{\PYZsq{}}\PY{l+s+s1}{Horas Diárias de Estudo: }\PY{l+s+s1}{\PYZob{}}\PY{l+s+s1}{str(df[}\PY{l+s+s1}{\PYZdq{}}\PY{l+s+s1}{Horas Diárias de Estudo}\PY{l+s+s1}{\PYZdq{}}\PY{l+s+s1}{].mean())[:5]\PYZcb{}}\PY{l+s+s1}{\PYZsq{}}\PY{p}{)}
          \PY{n+nb}{print}\PY{p}{(}\PY{n}{f}\PY{l+s+s1}{\PYZsq{}}\PY{l+s+s1}{Desempenho na Disciplina: }\PY{l+s+s1}{\PYZob{}}\PY{l+s+s1}{str(df[}\PY{l+s+s1}{\PYZdq{}}\PY{l+s+s1}{Desempenho na Disciplina}\PY{l+s+s1}{\PYZdq{}}\PY{l+s+s1}{].mean())[:5]\PYZcb{}}\PY{l+s+s1}{\PYZsq{}}\PY{p}{)}
\end{Verbatim}


    \begin{Verbatim}[commandchars=\\\{\}]
Horas Diárias de Estudo: 3.055
Desempenho na Disciplina: 74.66

    \end{Verbatim}

    Mediana:

    \begin{Verbatim}[commandchars=\\\{\}]
{\color{incolor}In [{\color{incolor}370}]:} \PY{n+nb}{print}\PY{p}{(}\PY{n}{f}\PY{l+s+s1}{\PYZsq{}}\PY{l+s+s1}{Horas Diárias de Estudo: }\PY{l+s+s1}{\PYZob{}}\PY{l+s+s1}{df[}\PY{l+s+s1}{\PYZdq{}}\PY{l+s+s1}{Horas Diárias de Estudo}\PY{l+s+s1}{\PYZdq{}}\PY{l+s+s1}{].median()\PYZcb{}}\PY{l+s+s1}{\PYZsq{}}\PY{p}{)}
          \PY{n+nb}{print}\PY{p}{(}\PY{n}{f}\PY{l+s+s1}{\PYZsq{}}\PY{l+s+s1}{Desempenho na Disciplina: }\PY{l+s+s1}{\PYZob{}}\PY{l+s+s1}{df[}\PY{l+s+s1}{\PYZdq{}}\PY{l+s+s1}{Desempenho na Disciplina}\PY{l+s+s1}{\PYZdq{}}\PY{l+s+s1}{].median()\PYZcb{}}\PY{l+s+s1}{\PYZsq{}}\PY{p}{)}
\end{Verbatim}


    \begin{Verbatim}[commandchars=\\\{\}]
Horas Diárias de Estudo: 3.0
Desempenho na Disciplina: 80.0

    \end{Verbatim}

    Desvio padrão:

    \begin{Verbatim}[commandchars=\\\{\}]
{\color{incolor}In [{\color{incolor}371}]:} \PY{n+nb}{print}\PY{p}{(}\PY{n}{f}\PY{l+s+s1}{\PYZsq{}}\PY{l+s+s1}{Horas Diárias de Estudo: }\PY{l+s+s1}{\PYZob{}}\PY{l+s+s1}{str(df[}\PY{l+s+s1}{\PYZdq{}}\PY{l+s+s1}{Horas Diárias de Estudo}\PY{l+s+s1}{\PYZdq{}}\PY{l+s+s1}{].std())[:5]\PYZcb{}}\PY{l+s+s1}{\PYZsq{}}\PY{p}{)}
          \PY{n+nb}{print}\PY{p}{(}\PY{n}{f}\PY{l+s+s1}{\PYZsq{}}\PY{l+s+s1}{Desempenho na Disciplina: }\PY{l+s+s1}{\PYZob{}}\PY{l+s+s1}{str(df[}\PY{l+s+s1}{\PYZdq{}}\PY{l+s+s1}{Desempenho na Disciplina}\PY{l+s+s1}{\PYZdq{}}\PY{l+s+s1}{].std())[:5]\PYZcb{}}\PY{l+s+s1}{\PYZsq{}}\PY{p}{)}
\end{Verbatim}


    \begin{Verbatim}[commandchars=\\\{\}]
Horas Diárias de Estudo: 1.333
Desempenho na Disciplina: 17.5

    \end{Verbatim}

    Variância:

    \begin{Verbatim}[commandchars=\\\{\}]
{\color{incolor}In [{\color{incolor}374}]:} \PY{n+nb}{print}\PY{p}{(}\PY{n}{f}\PY{l+s+s1}{\PYZsq{}}\PY{l+s+s1}{Horas Diárias de Estudo: }\PY{l+s+s1}{\PYZob{}}\PY{l+s+s1}{str(df[}\PY{l+s+s1}{\PYZdq{}}\PY{l+s+s1}{Horas Diárias de Estudo}\PY{l+s+s1}{\PYZdq{}}\PY{l+s+s1}{].var())[:5]\PYZcb{}}\PY{l+s+s1}{\PYZsq{}}\PY{p}{)}
          \PY{n+nb}{print}\PY{p}{(}\PY{n}{f}\PY{l+s+s1}{\PYZsq{}}\PY{l+s+s1}{Desempenho na Disciplina: }\PY{l+s+s1}{\PYZob{}}\PY{l+s+s1}{str(df[}\PY{l+s+s1}{\PYZdq{}}\PY{l+s+s1}{Desempenho na Disciplina}\PY{l+s+s1}{\PYZdq{}}\PY{l+s+s1}{].var())[:7]\PYZcb{}}\PY{l+s+s1}{\PYZsq{}}\PY{p}{)}
\end{Verbatim}


    \begin{Verbatim}[commandchars=\\\{\}]
Horas Diárias de Estudo: 1.777
Desempenho na Disciplina: 306.25

    \end{Verbatim}

    Mais informações sobre o dataset:

    \begin{Verbatim}[commandchars=\\\{\}]
{\color{incolor}In [{\color{incolor}375}]:} \PY{n}{df}\PY{o}{.}\PY{n}{info}\PY{p}{(}\PY{p}{)}
\end{Verbatim}


    \begin{Verbatim}[commandchars=\\\{\}]
<class 'pandas.core.frame.DataFrame'>
RangeIndex: 9 entries, 0 to 8
Data columns (total 3 columns):
Estudante                   9 non-null int64
Horas Diárias de Estudo     9 non-null float64
Desempenho na Disciplina    9 non-null int64
dtypes: float64(1), int64(2)
memory usage: 296.0 bytes

    \end{Verbatim}

    ~

    Referências Bibliográficas

    POLAMURI, Saimadhu. ``Data Mining with Python: Implementing
Classification and Regression''. Packt Publishing Company (2016)

    https://www.eecis.udel.edu/\textasciitilde{}portnoi/classroom/prob\_estatistica/2006\_2/lecture\_slides/aula20.pdf

    http://statisticsbyjim.com/glossary/regression-coefficient/

    BURGER, Scott V. ``Introduction to Machine Learning with R''. O'Reilly
Media (2018)


    % Add a bibliography block to the postdoc
    
    
    
    \end{document}
